%% Load document class fithesis2
%% {10pt, 11pt, 12pt}
%% {draft, final}
%% {oneside, twoside}
%% {onecolumn, twocolumn}
\documentclass[10pt,draft,oneside]{fithesis2}

%% Basic packages
\usepackage[english,czech]{babel}
\usepackage{cmap}
\usepackage[T1]{fontenc}
\usepackage{lmodern}
\usepackage[utf8]{inputenc}
\usepackage{graphicx}

%% Additional packages for colors, advanced
%% formatting options, etc.
\usepackage{color}
\usepackage{microtype}
\usepackage{url}
\usepackage{cslatexquotes}
\usepackage{fancyvrb}
\usepackage[small,bf]{caption}
\usepackage[plainpages=false,pdfpagelabels,unicode]{hyperref}
\usepackage[all]{hypcap}

%% Fix long URLs in DVIs
\usepackage{ifpdf}

\ifpdf
\else
  \usepackage{breakurl}
\fi

%% Packages used to generate various lists
\usepackage{makeidx}
%\makeindex

\usepackage[xindy]{glossaries}
%\makeglossary

%% Use STAR and CIRCLE signs for nested
%% itemized lists
%\renewcommand{\labelitemii}{$\star$}
%\renewcommand{\labelitemiii}{$\circ$}

%% \TODO příkaz
%\usepackage{xcolor}
\newcommand\fixme[1]{\textcolor{red}{[[#1]]}}
\newcommand\todo[1]{\textcolor{blue}{[[#1]]}}
% \renewcommand\todo[1]{}

\usepackage{natbib}             % sazba pouzite literatury
%\DeclareUrlCommand\url{\def\UrlLeft{<}\def\UrlRight{>}\urlstyle{tt}}  %rm/sf/tt
%\renewcommand{\emph}[1]{\textsl{#1}}    % melo by byt kurziva nebo sklonene,

%% Title page information
\thesistitle{Srovnání semiempirických metod EEM a QEq pro výpočet nábojů v molekulách}
\thesissubtitle{Bakalářská práce}
\thesisstudent{Jiří Daněk}
\thesiswoman{false} %% Important when using Slovak or Czech lang
\thesisfaculty{fi}  %% {fi, eco, law, sci, fsps, phil, ped, med, fss}
\thesislang{cs}     %% {en, sk, cs}
\thesisyear{jaro 2014}
\thesisadvisor{RNDr. Radka Svobodová Vařeková, Ph.D.}

%% Beginning of the document
\begin{document}
\selectlanguage{czech}

%% Front page with a logo and basic thesis information
\FrontMatter
\ThesisTitlePage

%% Thesis declaration (required)
\begin{ThesisDeclaration}
  \DeclarationText
  \AdvisorName
\end{ThesisDeclaration}

%% Thanks (optional)
\begin{ThesisThanks}
Rád bych na tomto místě poděkoval absolventu Fakulty informatiky Mgr. Stanislavu Filipčíkovi, autorovi šablony fithesis2 pro sazbu závěrečné práce v systému \LaTeX{}.

Řešení bakalářské práce mi značně zjednodušila možnost provést časově náročné výpočty v programu Gaussian 2008 na počítačích zpřístupněných v rámci Národní gridové infrastruktury MetaCentrum.

V neposlední řadě chci také poděkovat svým rodičům za podporu ve studiu.
\end{ThesisThanks}

%% Abstract (required)
\begin{ThesisAbstract}
Parciální náboje charakterizují elektronovou strukturu molekuly jen velmi omezeně, poskytují však cenné kvalitativní informace o chemických, fyzikálních i biologických vlastnostech molekul a mají využití i v kvantitativních modelech, jako je molekulová dynamika nebo predikce pKa. Pro výpočet parciálních nábojů bylo zavedeno vícero definic postihujících elektronovou strukturu molekul různým způsobem. Výpočet nábojů přímo dle definice často vyžaduje použití výpočetně náročných ab-initio metod, a proto se využívá i méně přesných empirických metod, které na základě parametrů získaných z dříve provedených ab-initio výpočtů aproximují parciální náboje s výrazně menší výpočetní náročností. Tato bakalářská práce popisuje a srovnává dvě empirické metody z hlediska jejich schopnosti reprodukovat Mullikenovy parciální náboje, EEM (Electronegativity equalization method) a QEq (Charge equilibration).

Přínos práce spočívá v implementování solverů pro EEM i QEq pod svobodnou licencí, ve vytvoření ucelené databáze dříve publikovaných sad parametrů pro metodu QEq a ve vyhodnocení přesnosti obou studovaných metod a všech shromážděných sad parametrů na rozsáhlé testovací sadě \fixme{1000} organických molekul.

\selectlanguage{english}
% http://tex.stackexchange.com/questions/24066/start-new-chapter-on-same-page
\begingroup
\let\clearpage\relax
\vskip 1cm plus 5mm
\chapter*{Abstract}
\endgroup

Partial charges, or net atomic charges, are only a crude description of the electronic structure of a molecule. Nevertheless, they are an useful characteristic of a molecule that can provide qualitative insights into physical, chemical and biological properties of a compound and can also serve as an input for quantitative computational models, be it Molecular Dynamics simulations or pKa prediction. There are several definitions of partial atomic charges that all reflect the electronic structure of the molecule in different ways and are therefore are incompatible with each other. Some definitions can be straightforwardly transformed into an ab-initio computation, which is usually impractically slow. Many empirical methods have been subsequently developed, each aims to provide results close to a particular ab-initio metod but in significantly shorter time. This thesis concerns itself with characterization and evaluation of two such empirical methods that have been developed for estimating Mulliken partial charges, namely EEM (Electronegativity equalization method) and QEq (Charge equilibration).

This thesis improves on the previously published research by compiling a comprehensive list of previously published parameter sets for the QEq method and by providing an evaluation of the accuracy of both EEM and QEq methods on a large test set of 1000 species of inorganic molecules. EEM and QEq solvers developed while working at this thesis are released under an open-source licence.
\end{ThesisAbstract}

\selectlanguage{czech}
\begin{ThesisKeyWords}
Huckellovy parcialní náboje, Electronegativity equalization method, Charge equilibration

\selectlanguage{english}
\begingroup
\let\clearpage\relax
\vskip 1cm plus 5mm
\chapter*{Keywords}
\endgroup

Huckell partial charges, net atomic charges, Electronegativity equalization method, Charge equilibration
\end{ThesisKeyWords}

\MainMatter

\selectlanguage{czech}

%U diplomky se MUSÍ vejít na jednu stranu.
\tableofcontents

%% Thesis text structured using
%% chapters, sections, subsections, etc.
\chapter*{Úvod}

Předmětem studia chemie je elektronový obal atomů, zejména jejich valenční elektrony, neboť ty se účastní tvorby chemických vazeb. Atomové jádro se nachází hluboko uvnitř elektronového obalu a z hlediska chemických vlastností atomu hraje jen malou roli (tím, jak ovlivňuje elektronový obal). Vlastnosti molekul jsou taktéž dány elektronovou konfigurací molekul. Můžeme tedy říct, že popis elektronové hustoty je středobodem chemie. \fixme{zdroj} V chemii existuje vícero způsobů, jak zachytit chování molekuly. Úplným popisem systému je vlnová funkce (viz xtý postulát). Od 70 let DFT, pro znalost velkého množsví měřitelných údajů není nutné znát vlnovou funkci, ale snačí znát distribuční funkci elektronové hustoty. Tu je jednodušší spočítat. Ze znalosti vlnové funkce je možno určit distribuční funci elektronové husty, ale ne naopak \todo{to je myslím Leach nebo to co cituju u toho I, II, III, ... Essentials cosi.} Ještě dalším zjednodušením proti DFT jsou parcialní náboje, které popisují elektronovou hustotu pomocí bodových nábojů umístěných v centrech atomů. Takové zjednodušení nutně vede ke ztrátě informace (existují distribuce elektronové hustoty, které není možno přesně popsat pomocí nábojů umístěných v centrech atomů, různé distribuce elektronové hustoty mohou vést ke stejným parcialním nábojům, neexistuje shoda na metodě, jak rozdělit elektronovou hustotu k nábojům), přesto ale mohou být užitečné.


Tato práce popisuje a srovnává dvě empirické metody pro určení

Prezentuje sady parametrů pro QEq v publikované literatuře. Srovnání metod mezi sebou a s ab-initio metodou na základě které byly parametrizovány na vícero skupinách molekul. Pro různé sady molekul. 

V této práci jsem implementoval dvě empirické metody pro stanovení parcialních nábojů atomů a srovnal je z hlediska rychlosti a kvality výsledku s přístupy Ab initio. EEM a Qeq

posuzuje metody dvě empirické metody pro výpočet nábojů v molekulách EEM a Qeq z hlediska rychlosti, se soustředí na empirické metody pro rychlý výpočet parcialních nábojů atomů. Obě metody jsou nejprve uvedeny do širšího kontextu. Srovnání je provedeno na sadě "Drug targets" Welcome Trust. 

Tento problém řeší empirické metody, které se snaží kopírovat vý dávat výsledky metody podle které byly parametrizovány za výrazně menší výpočetní náročnost.

Práce je členěna následujícím způsobem. První kapitola seznamuje čtenáře se základy počítačové chemie. Ve druhé kapitole je představen problém stanovéní parcialních nábojů a metody k jeho řešení. Jsou vysvětleny principy metod EEM a Qeq. Třetí kapitola shrnuje dosavadní výsledky diplomových prací a vědeckých článků na toto téma. Čtvrtá kapitola představuje moji implementaci obou metod a . Pátá kapitola obsahuje srovnání metod EEM a Qeq na testovací sadě molekul. Šestá kapitola shrnuje dosažené výsledky.

%This thesis is structured as follows. The problem of partial atomic charge estimation is first introduced assuming no computational chemistry experience. The principle of EEM and Qeq is explained. Both methods are implemented, parameter sets from previous published literature are compilled. Methods are evaluated by comparing the predicted charges with a gold standard of computed using the ab-initio method on Test set of drug published by the Welcome Trust regarding \todo{accuracy} using several statistical measures. The merits of this thesis of the published QEq parameter set and in evaluation of both methods with all parameter sets on a large set of 1000 inorganic molecules.

\chapter{Základní pojmy počítačové chemie}

Předmětem počítačové chemie (chemoinformatika, méně často cheminformatika) je využití počítačů v chemii. To obnáší jak digitální archivaci výsledků chemického výzkumu a následné vyhledávání v těchto datech, tak provádění výpočtů a simulací chemických a biochemických systémů s cílem formulovat nové hypotézy. Příkladem z první oblasti jsou databáze chemických sloučenin, reakcí, a faktografických údajů, které začaly vznikat od 60. let 20. století v reakci na v té době začínající exponencialní nárůst množství chemických poznatků. Do té doby narůstal počet známých chemických sloučenin přibližně lineárně. Když se tento trend změnil v exponenciální, chemici velmi rychle spatřili potenciál výpočetní techniky umožnit jim se s touto informační explozí vypořádat. (leech nebo comp chem, ta následujicí citace asi stačí) Příkladem z druhé, mladší oblasti počítačové chemie, která se někdy nazývá výpočetní chemie (computational chemistry) jsou metody molekulové mechaniky a molekulové dynamiky, které umožňují studovat chování molekul in-silico. \fixme{leach} Definic výpočetní chemie existuje mnoho (ta webstránka), autorem následující je Počítačová chemie usiluje o to, popsat všechny aspekty chemie v co nejlepší shodě s realitou pomocí výpočtů, raději než experimentálně. Attempts to model all aspects of real chemistry as closely as possible by using calculations rather than experiment. https://www.theochem.kth.se/people/ivo/philo.html

Chemoinformatika může být použita k předpovídání vlastností molekul, její předpovědi a simulace je ale nutno nakonec vždy ověřit experimentálně. Hlavní přínos chemoinformatiky spočívá v tom, že dovoluje výzkumníkům zaměřit se na perspektivní experimenty, když před tím rychle a levně zamítne ty neperspektivní. (zas historie comp chem) Konkrétním příkladem je virtualní screening, kdy se pomocí výpočetních metod vybírají kandidátské molekuly z rozsáhlé knihovny a zamítají se takové, které například mají nevhodné pH. \todo{nějaký článek, konkrétní příklad?}

\section{Reprezentace molekuly v počítači}

Předmětem zájmu počítačové chemie je molekula. V počítači můžeme molekulu reprezentovat buďto z hlediska její struktury, nebo geometrie.

\subsection{Strukturní reprezentace}

Molekula sestává z atomů, mezi kterými jsou chemické vazby. Strukturní reprezentace molekuly je z formálního hlediska vrcholově ohodnocený pseudograf. Pseudograf je multigraf, který může obsahovat smyčky (hrany vycházející i vcházející do stejného vrcholu). Multigraf je graf, mezi jehož dvěma vrcholy může být i více než jedna hrana. Vrcholy v pseudografu představují atomy v molekule, hrany v grafu představují chemické vazby mezi atomy, vrcholové ohodnocení grafu nám umožňuje přiřadit vrcholům chemické symboly prvků. Možnost vložit mezi dvojici vrcholů více než jednu hranu se využije pro popis vícenásobných vazeb. Možnost přidávat smyčky slouží k zachycení volných elektronových párů na atomech.

Formálně je struktura molekula popsána pseudografem $G = (V, E, L, j, b)$, kde $V$ je množina vrcholů, $E$ je multimnožina hran, $L$ multimnožina smyček a $j$ je funkce, přiřazující prvku množiny $V$ ohodnocení z množiny $b$.

Grafová reprezentace molekuly odpovídá strukturnímu vzorci, běžně používanému zejména v organické chemii. Využití grafů k reprezentaci molekul otevírá prostor využití grafových algoritmů k řešení chemických problémů. Tím nejzákladnějším, abychom byly schopni s molekulami v počítači pracovat, je jejich ekvivalence. Zjistit, zda dva grafy reprezentují stejnou molekulu znamená řešit problém izomorfismu grafů. Pro obecný graf je tento problém NP-úplný (citovat to ze skieny?), izomorfismus planárních grafů je možno řešit v čase O(n log n) a je možné využívat heuristiky, které dokáží dramaticky snížit časovou složitost průmětného případu. Molekulové databáze také mohou předpočítat indexy zrychlující následné dotazy.

\subsection{Geometrie molekuly}

Moderní experimentální metody, jako je Roentgenová krystalografie, dovolují  strukturou se rozumí prostorové uspořádání molekul. Tento zápis je při experimentů odhalujících strukturu molekul, jako je roentgenové krystalografie a kvantová mechanika.

Zápis molekuly její geometrií zanedbává informaci o vazbách a popisuje pouze pozice atomů v prostoru. Pozice atomů mohou být udány buď v kartézských souřadnících vůči libovolně zvolenému počátku souřadnic, nebo ve vnitřních souřadnicích, pomocí vzdáleností a úhlů vůči dříve uvedeným atomům.

Zadání v kartézské soustavě souřadnic znamená o každém atomu uvést jeho typ a trojici souřadnic x, y, z.

Zápis v interních souřadnicích se provádí pomocí Matice Z. U prvního atomu je uveden pouze jeho typ. Druhý atom je zadán typem a vzdáleností od prvního. Ke třetímu atomu je stačí uvést kromě typu ještě vzdálenost od druhého atomu a úhel mezi prvním, druhým a třetím atomem. Čtvrtý atom (a každý další) je určen typem, vzdáleností od předchozího atomu, úhlem mezi ním a předchozími dvěma a torzním úhlem rovin, kde první rovina je určených předchozími třemi atomy a druhou rovinu zadává poslední atom spolu s předchozími dvěma. Pro popsání obecné $N$-atomové molekuly potřebujeme $3N-6$ souřadnic. Číslice 6 se vztahuje k šesti nespecifikovaným stupňum volnosti: translaci a rotaci celé molekuly ve trojrozměrném prostoru.

K základním algoritmům pro práci s geometrickou reprezentací molekuly patří převod mezi interními a kartézskými souřadnicemi, přikládání molekulárních geometrií k sobě tak, aby mezi nimi byla co největší shoda. Počítačová grafika nachází aplikace při vizualizaci molekulových geometrií. Tyto vizualizace nahradili fyzické modely molekul.

\subsection{1D, 2D a 3D reprezentace molekuly v databázích}

1D daty se rozumí bibliografické informace o molekule, 2D data představují strukturní grafovou reprezentaci. 3D data geometrie molekuly.

\subsection{Souborové formáty}

Souborových formátů existuje v počítačové chemii velmi mnoho, protože autoři softwaru definovali vlastní pro každý software. Z vlastní zkušenosti jsem nabyl dojmu, že nejpoužívanější formáty jsou SDF, PDB, XYZ a SMILES.

Formít PDB (Protein Data Bank) http://www.wwpdb.org/documentation/file-format MOL/SDF (Structure-Data File) uchovává molekuly ve strukturní reprezentaci. http://cactus.nci.nih.gov/SDF_toolkit/ Molecular Design Limited. SDF je rozšíření MOL které umožňuje přidat k molekule attributy. XYZ je vpodstatě výčet prvků společně s jejich kartézskými souřadnicemi v molekule. http://openbabel.org/wiki/XYZ_%28format%29

SMILES (SImplified Molecular Input Line Entry Specification) je . Nad ním je postaven dotazovací jazyk, který je pro molekulární struktury jako regulární výrazy pro textové řetězce.



Reprezentace molekul grafem znamená, že počítačová chemie používá grafové algoritmy. Problém určit, zda jsou atomy totožné. Jedná se o grafový isomorfizmus, což je NP problém. Indexování, snaha testovat jen prespektivní přiložení a rychle odmítat.

Formáty. OpenBabel, softwarová knihovna a sada utilit pro převod mezi nimi.

\section{Hyperplocha potencialní energie}



\section{Molekulové modelování}

molekuly jsou moelovány jako kuličky na pružinkách. silová pole, neumí popsat vznik a zánik chemických vazeb, je výpočetně méně náročná, než kvantová mechanika. Typické silové pole AMBER, které

\section{Kvantová mechanika}
Kvantová mechanika vychází se . Vlnová funkce je kompletní charakterizace kvantového systému. Na čase nezávislá shcrodingerova rovnice.

Přípustné stavy systému jsou vlastní funkce \Psi operátorové rovnice

H\Psi = E\Psi \fixme{haček nad H}

kde H je operátor celkové energie, a E je energie příslušného stavu. Základní stav systému je stav s nejmenší energií.

Operátory, jsou funktory, 

Schrodingerovu rovnici je možné analyticky řešit jen pro chemické systémy s jedním elektronem, to jest vodíkový atom (kationt He+). Bohr Oppenheimerova approximace, která

elektrony a jádra jsou bodové náboje
pozice jader je fixní a předem daná
elektrony jsou nezávislé, se navzájem neovlivňují

řeší se optimalizační problém nalezení takové \(\Psi\), vyjádřené jako linearní kombinace bázových orbitalů, aby E byla minimální.

Jako bázové orbitaly se používají jednoelektronové orbitaly získané analytickým řešením vodíkového atomu.

HF/3

metoda/báze

Limity aproximace

Korelační energie elektronů je značná

Basis Set Superposition Error

výsledek výpočtu závisí na zvolené bázi

levels of theory

empirické metody, které 

zanedbává korelační energii

program gaussian, leach s. 8


\chapter{Parcialní náboje}

Parcialní náboje jsou velmi starý chemický koncept, který popisuje distribuci elektronové hustoty v molekule pomocí bodových nábojů lokalizovaných na jádrech atomů. Parcialní náboje dovolují formulovat kvalitativní souvislosti některé vlastnosti molekuly jako celku a topologii jejích chemických vazeb. (dipólové momenty, ...) 

Pro jednotlivé volné atomy definujeme jejich elektronegativitu, realné číslo vyjadrující jejich schopnost přitahovat elektrony. Existuje vícero definic elektronegativity (Paulingova, experimentální měření, ...)  \todo{Naopak valenční elektrony .}

Mezi parcialním nábojem a elektronegativitou existuje přímá souvislost. Samostatné atomy jsou elektricky neutální. Při utváření molekulových vazeb atomy navázané na ty s vyšší elektronegativitou nabývají parcialní kladný náboj v důsledku ztráty valenčních elektronů ve prospěch svého vazebného partnera, zatímco atomy s méně elektronegativním vazebným partnerem od něj elektrony přebírají. \todo{toho Sandersona můžu dát už sem}

Parcialní náboje se definují pomocí

Na ekvalizaci náboje vznikla celá řada metod, 1985 EEM, 1991 QEq, SQE pro rychlý výpočet náboje, od každé z těchto metod dále existují odvozené metody. Hodnocení korelace. Parametrizují se Mullikenovy, DFT, nějaké další (ty benchmarky), úspěšnost se hodnotí std odchylkou, korelací. Parametrizují se i parametry molekul, dipolové momenty, 

Jednou z metod stanovéní parcialních nábojů biomolekul je a fitování podle experimentu. Metoda . Okolo molekuly se vytvoří myšlená mřižka a metodou nejmenších čtverců se stanoví náboje na atomech tak, aby reprodukovaly experimentálně naměřené hodnoty v bodech této mřížky.

mřížka je citlivá na rozmístění bodů? že se naučí mřížku a ne molekulu, resp distribuci náboje kolem molekuly, přeučení? nebo že různá mřížka, různé výsledky?

Náboje na atomech uvnitř molekuly nejsou významné a metoda má tendence stanovit je nerealisticky obrovské.

R  Restricted přidává další omezení aby se předešlo výše uvedeným problémům.



\section{Ab initio, semiempirické a empirické metody}

Latinská fráze \emph{Ab initio}, která se překládá jako „z prvotních principů“, v kontextu výpočetní chemie označuje výpočetní metody, které vycházejí z kvantové mechaniky. Jedním z postulátů kvantové mechaniky je, že vlnová funkce je úplnou charakteristikou systému. Ab initio metody fungují na principu určení vlnové funkce a z ní následně žádaných veličin. Někteří autoři do této skupiny zahrnují i metody založené na funkcionálu elektronové hustoty (DFT,  \textit{\foreignlanguage{english}{Density Functional Theory}}).

DFT metody jsou co se týče principu výpočtu velmi podobné ab initio metodám. Namísto vlnové funkce se pro charakteristiku systému používá funkcionál elektronové hustoty, který sice není úplnou charakteristikou systému, pro mnohé aplikace ale plně dostačuje, a navíc je jeho výpočet realizovatelný i pro velké biomolekuly, pro které by určení plné vlnové funkce trvalo příliš dlouho.

Semiempirické metody jsou modifikací ab initio metod takovou, že tam, kde to výrazně přispěje ke zrychlení výpočtu, se za cenu nižší přesnosti použije předem určených parametrů, které mohou být zjištěny i experimentálně.

Empirické metody staví na jiných teoretických základech, než je kvantová mechanika. Příkladem může být molekulová mechanika, technika pro určování konformace a dalších parametrů molekul ze zjednodušeného modelu molekuly, který je založený na zákonech klasické fyziky (chemická vazba se například reprezentuje jako pružina). Využívání experimentálních parametrů je u empirických metod téměř pravidlem.

Jedno z kritérií kvality semiempirických a empirických metod se nazývá přenositelnost parametrů. V praxi se ukazuje, že v rámci jednotlivých tříd strukturně podobných molekul bývají empirické parametry pro jednotlivé molekuly velmi podobné, a proto je možné stejnou parametrizaci použít pro výpočty nad všemi molekulami dané třídy. Čím je míra přenositelnosti parametrů vyšší, neboli čím širší škálu molekul můžeme dostatečně přesně popsat s použitím jedné parametrizace, tím lépe vyhodnocovaný model generalizuje realitu.

%Klasickou Newtonovskou mechaniku je možno označit za empirickou teorii. Gravitační konstanta je empirický parametr. Relativistická fyzika: tuto konstantu vypočítat.

%An alternative measure of the charge distribution involves a partitioning into partial atomic charges. While such partitioning is always arbitrary (see Chapter 9) simple methods tend to

%Valence orbitals, on the other hand, can vary widely as a function of chemical bonding. Atoms bonded to significantly more electronegative elements take on partial positive charge from loss of valence electrons, and thus their remaining density is distributed more compactly. The reverse is true when the bonding is to a more electropositive element. From a chemical

%Highly charged regions of molecule are the most reactive locations
%Charges = clue to reactivity
%Charges provide a very useful information about a molecule
%Charges provide a deep insight into a chemical behaviour of a molecule


\todo{vyuziti naboju k fitovani vlastnosti}
%Charges are excellent descriptors in QSPR and are necessary in simulations = molekulová mechanika
%predikce fyzických, chemických a biologických vlastnosti
% a jsou parametrem vstupem mnoha výpočetních modelů na předpovídání

%výpočet stability molekul

%odhad směru průběhu chemické reakce

%předpovědi interakcí s dalšími molekulami

\todo{parcialní náboje mají fyzikální smysl}

\todo{experimentalni dukazy spektroskopie}

\todo{polarita molekul}

\todo{dipolove momenty, ...}

%značně nepřesná charakterizace

%Parametrizace EEM pro biomolekuly

%Roentgenová krystalografie

Rozložení elektronové hustoty v molekule je pozorovatelná veličina. Existuje pro ně kvantový operátor a možno pozorovat například pomocí Roentgenové krystalografie. je možno měřit, přiřazení elektronů k atomům je pouze zjednodušující popis realné distribuce elektronové hustoty. Elektronová hustota je rozložena mezi atomy, nikoli na atomech a neexistuje experimentální metoda pro určení nábojů na atomech.

\chapter{Parcialní náboje}

Ve stejnojaderných dvouatomových molekulách působí obje atomová jádra na elektrony stejně a elektronová hustota (vazebných elektronů) je taktéž rovnoměrně rozdělena mezi obje jádra. Vazby, podél kterých je elektronová hustota rozdělena rovnoměrně se nazývají nepolární. Dvouatomové molekuly složené z různých atomů mají elektronovou hustoru rozloženu podél vazby nesymetricky, protože jeden z atomů (v důsledku rozdílného protonového čísla a elektronové konfigurace) přitahuje elektrony větší silou, než ten druhý. Takové vazby se nazývájí polární, nebo, pokud jsou elektrony velmi silně přitahovány k jenomu z atomů, iontové. Ve víceatomových molekulách dochází k přesunu elektronů přes více než jednu vazbu. Vysoce elektronegativní substituenty v molekulách derivátů uhlovodíků (například halogeny) snižují elektronovou hustotu na navázaném uhlíkovém skeletu. S rostoucí vzdáleností od substituovaného atomu se účinek tohoto efektu snižuje. odmaturujZChemie s 120

Elektronová hustota se nachází mezi atomy tvořícími vazebný pár, nikoli na atomech. Přiřazení elektronové hustoty k atomům, ač technicky ne zcela přesné, se ukázalo být velmi praktické, jak už při zápisu vzorců, tak při výpočtech.

Experimentálně měřitelným projevem nerovnoměrného rozložení elektronové hustoty v molekule je dipólový moment molekuly. Permanentní dipól molekuly se projeví ve spektru získaném mikrovlnnou spektroskopií. Příkladem z praktického života je ohřev vody v mikrovlné troubě, kdy magnetické pole trouby způsobí rotaci polárních molekul vody. Dipól dvouatomové molekuly s parcialními náboji $-q$ a $q$ ve vzálenosti $r$ je $\mu = qr$. Dipól obecné molekuly je vektorový součet $\mu = \sum_j q_j p_j$, kde $q_j$ je náboj a $p_j$ je vektor souřadnic $j$-té molekuly.

Schopnost atomu v molekule přitahovat elektrony zachycuje veličina zvaná atomová elektronegativita, kterou zavedl Linus Pauling a bývá značená obvykle $X$. Metod pro její výpočet existuje několik, například $X = k(I+A)$, kde $I$ je ionizační energie a $A$ je elektronová afinita příslušného atomu. Autorem právě uvedené definice je R. S. Mulliken. Elektronegativita je relativní veličina a je vztažena ke zvolenému referenčnímu prvku, kterým bývá zpravidla fluor. prehledChemie s 104 atkins 380

Parcialní náboje jsou reálná čísla, která popisují podíl elektronové hustoty příslušející k jednotlivým atomům v molekule. Z této definice vyplývá, že parcialní náboje jsou experimentálně neměřitelný teoretický koncept, protože ačkoli elektronovou hustotu je možno přesně změřit například pomocí rentgenové krystalografie nebo ji vypočíst pomocí ab initio metod, schéma jejího přiřazování k atomům musí být nutně arbitrární. Pro výpočet parcialních nábojů bylo zavedeno mnoho metod, z nichž každá přiřazuje elektronovou hustotu k atomům jiným způsobem, a proto také dávají navzájem rozdálné číselné výsledky.

V chemické teorii se běžně nepracuje s absolutními hodnotami parcialních nábojů, ale s jejich rozdíly a trendy. \fixme{zdroj?} Pomocí rozdíů v hodnotách parcialních nábojů se například v organické chemii vysvětlují jevy jako vodíkové vazby nebo reaktivita funkčních skupin (pojmy jako indukční a mezomerní efekt). \fixme{zdroj: přehled sš chemie, vodíkové vazby i atkins} Význam parcialních nábojů v chemii je tak především kvalitativní.

Parcialní náboje je možno používat i kvantitativně, například jako jeden ze vstupů regresního modelu k predikci disociačních konstant. V tom případě je ale vždy dáno, jakou metodou je nutno náboje počítat.

%Problém je s přiřazením náboje k jednotlivým atomům. K tomu existují různé metody, které dávají různé výsledky.

%Příčinou vzniku atomových nábojů je nerovnoměrné rozložení elektronů v molekule.

%, úzce souvisí s rozložením elektronové hustoty v molekule poskytují celou řadu užitečných informací. Simulace

Kritériem správnosti kvantitativních výstupů jsou užitečnost a obecnost výpočetní metody. Užitečností se myslí existence experimentu, s nímž výpočetní metoda vykazuje dobrou, jinými slovy, který dobře modeluje. Obecnost je parametrem u empirických metod a jedná se primárně o míru přenositelnosti parametrů.

Ekvalizace elektronegativity

Elektronegatita je realné číslo obvykle v rozsahu \todo{rozsah} Je možno chápat jako schopnost přitahovat elektrony. Při pohledu na periodickou tabulku platí, že při pohybu směrem doprava a nahoru elektonegativita roste. Existuje mnoho schémat pro výpočet elektronegativity. Elektronegativita se definuje pro volné atomy.



Vznik molekuly spočívá ve sdílení elektronů mezi vícero atomy. Ustavení molekuly můžeme chápat jako ustavení rovnovážného stavu v uspořádání elektronového oblaku. Pokud by se elektonegativita zvoleného atomu zvýšíla, bude více obklopen elektrony, což jeho elektronegativitu opět sníží.

\section{Metody výpočtu}

Metody pro výpočet nábojů se dají rozdělit do čtyř kategorií. \citep[s.~50]{cramer2004essentials}

Třída I představuje empirické metody, které vycházejí nikoli z kvantové mechaniky, ale jsou založeny na fyzikálních analogiích a intuici tvůrců. Tyto metody mohou využívat experimentálních dat, jako jsou dipólové momenty nebo elektronegativity.

\todo{definice elektonové hustoty a vzorec?}

Třídy II a III zahrnují metody, které vycházejí buď přímo z vlnové funkce (třída II), nebo z pozorovatelné veličiny z vlnové funkce vypočtené (třída III), například z elektronové hustoty, a na základě intuitivně zvoleného schématu ji rozdělí na příspěvky od jednotlivých atomů.

Třída IV je vyhrazena metodám, které vycházejí z metod ve třídách II a III, rozdělení nábojů ale hledají takové, které nejlépe odpovídá experimentálně určeným parametrům, například dipólovému momentu molekuly.

V praxi se nejčastěji používají hodnoty parcialních nábojů získané metodami ve třídách II a III. V případě, že pro požadovanou aplikaci je výpočet některou z metod tříd II nebo III příliš časově náročný, je možné náboje aproximovat použitím vhodné metody z třídy I.

%Výsledkem této práce je 
%Ačkoli jsou atomové parcialní náboje v principu neměřitelný teoretický konstrukt, zůstanou i na dále důležitým prvkem v chemii. Do způsobů, jak je určovat bylo investováno mnoho výzkumného úsili.

%Polární molekuly

%Nepolární molekuly

%Molekuly léčiv


Metody equilibrace náboje jsou motivovány vyjádřením potencialní energie atomu v závislosti na náboji a minimalizováním potencialní energie za podmínky známé hodnoty celkového náboje molekuly.

Elektronegativitu můžeme chápat ionizačního potencálu a elektronové afinity izolovaných atomů, nebo jako další parametry.

\section{Electronegativity Equalization Method (EEM)}

V článku Electronegativity Equalization: Application and Parametrization



Metoda Electronegativity equalization

tří předpokladů

efektivní elektronegativita atomu v molekule je elektronegativita izolovaného atomu, snížená nebo zvýšená o

elektrony v molekule se přeskupí od méně elektronegativních atomů k více elektronegativním atomům tak, že efektivní elektronegativita všech atomů v molekule bude stejná.

Pro každý atom můžeme napsat rovnici ve tvaru

Všechny tři rovnice můžeme úsporně vyjádřit maticovým zápisem ve tvaru


EEM je první posuzovaná empirická metoda pro výpočet parcialních nábojů.

vychází ze tří podmínek. První podmínkou . Druhá podmínka, která dala metodě její jméno, je Sandersonův princip ekvalizace elektronegativity, a třetí podmínkou je zachování celkového náboje molekuly. První podmínka popisuje závislost potencialní energie molekuly na parcialních nábojích, na vzdálenosti atomů, jejich elektronegativitě \todo{a}. Za předpokladu, že studovaná molekula je v rovnovážném stavu můžeme usoudit, že se nachází ve stavu s nejnižší potencialní energií. Víme tedy, že deriace výrazu pro energii musí být nulová. Tímto krokem jsme odstranili energii jako neznámou a z tohoto vztahu si můžeme vyjádřit efektivní elektronegativitu každého z N atomů v molekule jako funkci jeho náboje. Druhá podmínka nám umožňuje vytvořit N-1 rovnic porovnávajících elektronegativitu. Třetí podmínka poskytuje poslední, N-tou, rovnici, kterou potřebujeme. \todo{jenomže rovnic je n+1}.

Nejenom z hlediska úspornosti zápisu, ale také pro pozdější počítačové řešení je vhodné tuto soustavu rovnic zapsat maticovým schématem

Aq=b, kde q je vektor neznámých.

B_1 \frac{\kappa}{R{1,2}} ... \frac{\kappa}{R{1,N}} -1
\frac{\kappa}{R{2,1}} B_2 ... \frac{\kappa}{R{2,N}} -1

\frac{\kappa}{R{N,1}} \frac{\kappa}{R{N,2}} ... B_N -1
1 1 ... 1 0

q_1
q_2
...
q_N
\bar{\xi}

-A_1
-A_2
...
-A_N
Q



Pro použití EEM potřebujeme znát prostorové uspořádání atomů v molekule, parametry a pro každý atom a konstantu kapa.

Potencialní energie

Můžeme značit E, neboť jiná než potencialní energie v tomto odvození nevystupuje nebo V (jak je obvyklé v kvantové mechanice)

$$E=\frac{q_1 q_2}{\epsilon r_{1,2}}$$

Parametrizace

Parametry mají chemický význam

\section{Charge Equilibration (QEq)}

Stejně jako u metody EEM vycházíme z výrazu pro potencialní energii. Autoří QEq ve svém odvození nezabývají tím, jak tento vztah vypadá, ale 

že dE/dq

pro molekulu potom


Použítím vztahu pro ekvalizaci elektronegatity, což je 

Coulumbův zákon neplatí pro velmi malé vzálenosti, kdy už nelze atomy realisticky aproximovat jako bodové náboje. Proto  $q_1 q_2 J_{1,2}$, J je coulombovský integrál. Pro velké hodnoty R J = 1/R což je Coulumbův zákon a pro malé hodnoty se používá překryvový integrál pro Slaterovy orbitaly Jij = .

Parametrizace

Stejně jako u metody EEM mají parametry chemický význam.

\section{Split Charge equilibration (SQE)}

Tato sekce začíná popisem split charges a následně popisuje odvození modelu SQE způsobem v originálním článku.

Na rozdíl od předchozích metod uvažuje SQE i topologii molekuly.

Definice split charge

\todo{Split charge byl zaveden v metodě AACT.}

Náboj atomu vyjádříme jako jeho vlastní náboj a přesunutý náboj z jeho sousedů po kovalentních vazbách (\todo{v učebnici chemie na to je možná nějaká terminologie}) součet "dělených" nábojů na v Split charge.

\[Q_i = \sum_j overbar qij\]

Odvození metody Split Charge equilibration

V původním článku vycházejí z ekvilibrace náboje, na čež zavedou split charges podle definice nahoře, napíší původní vstah z jejich pomocí, v dalším kroku zavedou nové proměnné a ukáží, že při zavedení dalších tří podmínek je jejich konstrukce ekvivalentní s ekvilibrací náboje. Při přidání jiných podmínek přechází v metodu AACT. Tak je ukázáno, že metoda SQE (kde se dodatečné podmínky nevyskytují) je zobecněním obou těchto metod.

Princip ekvilibrace náboje, ekvivalentně ekvilibrace elektronegativity

\todo{proč je tady náboj to samý co chemický potenciál?, molární g energie?}

\chapter{Předchozí implementace}
EEM

Programy implementující parametrizaci musejí implementovat i řešení EEM systému.

práce na MU

včetně parametrizace

Tomáš Raček ve své diplomové práci implementoval program NEEMP ()

Bakalářská práce Jakuba Vaňka prezentuje nástroj EMP (EEM Method Parametrization). EMP 

Poslední nástroj nazvaný TRON vytvořil Tomáš Raděj.

Řešení EEM systému

TRON a EMP používají vlastní implementaci Gaussovy eliminace s pivotováním, NEEMP používá \todo{ale co je to za konkrétní operace} procedury ssysv, sgesv a sspsv  z knihovny Intel Math Kernel Library.

Parametrizace

Tron používá exhaustivní prohledávání intervalu kappa, NEEMP vychází se stejného přístupu, s tím, že v první fázi prostupně projde interval kappa řídčeji s experimentálně určenou konstatnou a následně takto experimentálně zjištěnou závislost lokálně aproximuje parabolami a snaží se cíleně prozkoumávat lokální maxima je postavený na Brentově metodě využívá GNU Scientific Library. \todo{ten NEEMP je asi/možná spatně popsané}

EMP a NEEMP vyřazují některé molekuly z učící sady, což vede k vylepšení výsledné korelace na celé sadě.

Programovací jazyk. TRON je napsán v jazyce Java, NEEMP v jazyce C.

Licence. K programu TRON jsou dostupné zdrojové kódy.

Paralelní provádění

Tron a \todo{EMP} pracují seriově. NEEMP používá OpenMP a matematická knihovna interně používá OpenMP a \todo{SIMD} Míru paralelizmu je možno volit.

NEEMP dospěl k závěru, že paralelní výpočet matice není vhodný a paralelizmus co se týče výpočtů pro ruzné hodnoty lambda

bez parametrizace

Radka Svobodová Vařeková 

další sw

openbabel

QEQ

openbabel
qtpie
Materials Studio
GULP



SQE

vše možné
qfit, cit verstraelen2011significance , není dostupný

Hypotéza

Metoda nejmenších čtverců je známá tím, že malé množství odlehlých bodů může značně ovlivnit výsledek. Je tedy možné, že v učící sadě se nacházejí molekuly, které dají vzniknout odlehlým bodům v příslušném kroku parametrizace. Tyto molekuly mohou být buď neobvyklé z hlediska distribuce náboje a námi používaný model není schopen správně postihnout jak tyto body, tak většinu molekul. Vynecháním ze sady dovolíme modelu správně se naparametrizovat na většinu, což vede k lepší korelaci, než pokus parametrizovat vše, ale nic správně. Další možnost je, že topologie těchto molekul je v databázi zadána chybně. Dále je možné, že data jsou jinak poškozená. Chybějící atomy a tak podobně. Metoda EEM je odvozena pro molekuly v globálním minimu nebo blízko něj \todo{je to pravda?}. Není možné vyloučit ani kombinaci všech zmíněných možností.

Metoda

Hypotézu o chybných vstupních datech je možno ověřit pomocí optimalizace vstupních molekul. Pokud se optimalizovaná konfigurace liší od vstupní, znamená to, že molekula byla zadána chybně. Optimalizace je iterativní proces. Pro urychlení této operace stačí provést pouze první iteraci optimalizačního procesu, což odhalí, zda byla molekula v lokálním minumu nebo ne.

Všechny hypotézy je možno otestovat manualní inspekcí vyřazených molekul.

\chapter{Implementace}

\section{Electronegativity equalization method}

\section{Charge equilibration}

Metodu Charge Equilibration (QEq) v článku Charge Equilibration for Molecular Dynamics Simulations \fixme{citace}.

\chapter{Použité sady molekul}

DTP NCI Developmental Therapeutics Program

%\url{http://dtp.nci.nih.gov/docs/3d_database/dis3d.html}

Milne, G.W.A., Nicklaus, M.C., Driscoll, J.S., Wang, S. and Zaharevitz. D. The NCI Drug Information System 3D Database. J. Chem. Inf. Comput. Sci. 34:1219-1224 (1994).

\chapter{Výpočet parcialních nábojů QM}

Gaussian . Název je odvozen od Gaussových křivek, které se používají k aproximaci složek vlnové funkce.

%http://cactus.nci.nih.gov/download/nci/

BLAS (Basic Linear Algebra Subprograms) je API specifikující často používané operace linearní algebry nad vektory a maticemi. Pro přehlednost je knihovna rozdělena na tři části. BLAS Level 1 Blas Level 2 a Blas Level 3 obsahuje procedury pro operace mezi dvěma maticemi. Nad primitivy obsažené v BLAS je postaven LAPACK (Linear Algebra PACKage). První implementace v jazyce FORTRAN77 Pro obě rozhraní existuje celá řada implementací, výrobci počítačových procesorů často dodávají vlastní implementace, ATLAS obsahuje benchmark, který automaticky volí nejlepší implementace pro hardware. Obsahují významnou součást benchmarků pro HPC.

Jména procedur reflektují datový typ (s, single precision, float32, d, double precision, float64)







Pro řešený problém je význačný následujícími dvěma aspekty. Za prvé řešení soustav linearních rovnic, které je ale z hlediska řešeného problému implementační detail a pro jeho řešení je zapotřebí znalostí numerické matematiky, který je nejlépe nechat knihovnám.

Přepínače kompilátoru

kompilátory mají celou řadu přepínačů, které ovlivňují rychlost výsledného kódu. Všechny přepínače nejsou dostupné u všech kompilátorů.

ffast-math

vypíná kompatibilitu s IEEE755 či co. Kompilátor například (a * a * a * a) je vyhodnocováné jako (((a * a) * a) *a), tři násobení, s přepínačem je možno optimalizovat jako (a*a) * (a*a), dvě násobení. Citace?

-O1,O2, O3, Os

harden source

https://security.stackexchange.com/questions/24444/what-is-the-most-hardened-set-of-options-for-gcc-compiling-c-c
-fstack-protector-all -Wstack-protector --param ssp-buffer-size=4 
%-­D_FORTIFY_SOURCE=2 ­O2



Hardware

Lenovo Thinkpad E530 (X86\_64)
mobil! tablet! ARM
MetaCentrum
Aisa

Software

Kompatibilitu programu jsem ověříl překladem na třech různých na sobě nezávisle vyvíjených překladačích

g++ GCC Gnu Compiler Collection
Clang
icpc, kompilátor vyvíjený firmou Intel, verze pro Linux je k dispozici zdarma k nekomerčnímu využití.
android ndk
https://pm.bsc.es/projects/mcxx mercuium source to source

KDevelop

dovození typu (type inference), což umožňuje používat konstrukci auto. Další vývojová prostředí, která by splňovala moje požadavky QTCreator, Microsoft Visual Studio, Eclipse.

OpenBabel

Boost program options

Eigen3
\todo{kopírování, přímý BLAS dovoluje provádět více operací inplace?}

formát souboru s parametry

na základě formátů používaných ostatními programy

řádkový
oddělovač řádku může být cokoli běžně používané

\fixme{řádek #! jména sloupců, more trouble than it's worth}
řádky začínající # jsou komentáře \fixme{dovolit koment na konci řádku?}

není možné vynechat hodnotu ve sloupci
pořadí sloupců je dané výpočetní metodou
oddělovačem je mezera a tabulátor (nebo unicode whitespace)

je shopno postihnout i vstup pro SQF, kde je nutno zadávat i parametry pro vazby

klíč hodnota hodnota hodnota ...

stejný klíč není dovoleno uvést vicekrát

parametr pro zvalidování vstupu


\chapter{Parametrizace}

EEM: pro fixní kappa najdeme další parametry metodou nejmenších čtverců. Programy TRON, EMP a * používají tuto metodu.

QEq: Program v článku kadantsev2013fast používá kombinaci globální a lokální optimalizace, konkrétně vlastní genetický algoritmus následovaný gradientní metodou (podle informací v dodatku se jednalo o steepest descent) pro doladění parametrů. Optimalizují se pouze parametry elektronegativita a tvrdost, ostatní zůstávají konstantní.

\chapter{Vyhodnocení}

Existuje rozpor ohledně toho, co se od metod jako EEM a QEq učekává. Jako chemická teorie, parametry vypovídají o chemických vlastnostech zkoumaných systémů, nebo regresní/metoda strojového učení, která vytváří a výstup slouží pro další regresní modely předpovídající chemické vlastnosti. Ve druhém případě se nezajímáme o chemický význam parametrů, ale o kvalitu predikcí.

V této práci hodnotím metody z hlediska jejich schopnosti reprodukovat QM výpočet. Výsledné náboje pro použití do regresního modelu, tedy nezáleží ani tak na absolutní shodě, jako spíš na korelaci. (Relativní posun se regresní model může naučit kompenzovat).

Databáze

Mnohé informatické vědní obory mají tradici zkoušení metod na de facto standardních sadách testovacích dat.

\todo{SQE článek má sadu molekul.}

Je výhodné testovat různé pokusy o řešení problému na těch stejných datech.

Referencni databaze jsou dulezite z nasledujicich duvodu, poskytuji moznost srovnat tuzne implementace otestovane ruznymi lidmi, je na nich mozne sledovat vyvoj a zlepsovani metod v prubehu casu. Dobra referencni databaze by mela byt snadno dostupna, dostatecne variablilni, pripadne se zamerovat na zvlaste slozite instance, ale nesmi opomenout ani ty bezne

Databaze se muze delit na ucici a testovaci data, pripadne si toto rozdeleni provede az pri testovani algortmu jako proccentualni cast, nebo pomoci foldu a podobne.

Physionet, street nnumbers, computer vision competitions, machine learning ma nektere notoricky zname, east west train,

Nevýhodou je že postupem času autoři se mohou začít soustředit na vylepšení výkonu na testovacích sadách, namísto reality. Znají svoji sadu velmi dobře po letech práce a dělají věci, které přinesou pokrok jen pro řešení sady, ne pro praktickou aplikaci. Problém přeučení.

U modelu muzeme preferovat vypocetni jednoduchost, coz je pripad chemoinformmatiky, kde boovykle ypracovavame velke mnozstvi molekul a je vyhodne co nejvice kandidatu co nejrzchleji zamitnout a soustredit se jen na tz perspektivni. V pocitacovem videni je zadouci bzt schopen zpracovavat data v realnem case, tj 30 az v nekterzch aplikacich 120 i vice snimku za sekundu. Klasifikator vztvorenz pomoci boosting jednoduchzch klasifikatoru a kaskady s fast reject

\chapter{Závěr}

Ukazuje se, že obě metody jsou v dobré shodě s ab-initio výpočtem. Metoda QEq dosahuje výrazně lepších výsledků než EEM za cenu jen mírného zvýšení časové náročnosti výpočtu.

%% Lists of tables and figures, glossary, etc.
%\printindex
%\printglossary
%\listoffigures
%\listoftables

%% Bibliography from citace.bib
%\bibliographystyle{plain}
\bibliographystyle{csplainnat}
\bibliography{charges}

%% Additional materials
\appendix

Návod k programu Quick

%% End of the whole document
\end{document}
