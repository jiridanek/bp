%% Load document class fithesis2
%% {10pt, 11pt, 12pt}
%% {draft, final}
%% {oneside, twoside}
%% {onecolumn, twocolumn}
%%\documentclass[10pt,draft,oneside]{fithesis}

\documentclass[
  color, %% This option enables colorful typesetting. Replace with
         %% `monochrome`, if you are going to print the thesis on
         %% a monochromatic printer.
  table, %% Causes the coloring of tables. Replace with `notable`
         %% to restore plain tables.
  lof,   %% Prints the List of Figures. Replace with `nolof` to
         %% hide the List of Figures.
  lot,   %% Prints the List of Tables. Replace with `nolot` to
         %% hide the List of Tables.
  %% More options are listed in the class documentation at
  %% <http://mirrors.ctan.org/macros/latex/contrib/fithesis/fithesis/guide/mu/fi.pdf>.
]{fithesis3}

%% Basic packages
\usepackage[main=czech, english]{babel}
\usepackage{cmap}
\usepackage[T1]{fontenc}
\usepackage{lmodern}
\usepackage[utf8]{inputenc}
\usepackage{graphicx}

%% Additional packages for colors, advanced
%% formatting options, etc.
\usepackage{color}
\usepackage{microtype}
\usepackage{url}
\usepackage{cslatexquotes}
\usepackage{fancyvrb}
\usepackage[small,bf]{caption}
\usepackage[plainpages=false,pdfpagelabels,unicode]{hyperref}
\usepackage[all]{hypcap}

%% Fix long URLs in DVIs
\usepackage{ifpdf}

\ifpdf
\else
  \usepackage{breakurl}
\fi

%% Packages used to generate various lists
\usepackage{makeidx}
%\makeindex

%\usepackage[xindy]{glossaries}
%\makeglossary

%% Use STAR and CIRCLE signs for nested
%% itemized lists
%\renewcommand{\labelitemii}{$\star$}
%\renewcommand{\labelitemiii}{$\circ$}

%% \TODO příkaz
%\usepackage{xcolor}
\newcommand\fixme[1]{\textcolor{red}{[[#1]]}}
\newcommand\todo[1]{\textcolor{blue}{[[#1]]}}
% \renewcommand\todo[1]{}

\usepackage{natbib}             % sazba pouzite literatury
%\DeclareUrlCommand\url{\def\UrlLeft{<}\def\UrlRight{>}\urlstyle{tt}}  %rm/sf/tt
%\renewcommand{\emph}[1]{\textsl{#1}}    % melo by byt kurziva nebo sklonene,

%% The following section sets up the metadata of the thesis.
\thesissetup{
    university    = mu,
    faculty       = fi,
    type          = bc,
    author        = Jiří Daněk,
    gender        = m,
    advisor       = RNDr. Radka Svobodová Vařeková, Ph.D.
    title         = {Srovnání semiempirických metod EEM a QEq pro výpočet nábojů v molekulách},
    TeXtitle      = {Srovnání semiempirických metod EEM a QEq pro výpočet nábojů v molekulách},
    keywords      = {Huckellovy parcialní náboje, Electronegativity equalization method, Charge equilibration},
    TeXkeywords   = {Huckellovy parcialní náboje, Electronegativity equalization method, Charge equilibration},
}
%% Abstract (required)
%\begin{ThesisAbstract}
\thesislong{abstract}{
Parciální náboje charakterizují elektronovou strukturu molekuly jen velmi omezeně, poskytují však cenné kvalitativní informace o chemických, fyzikálních i biologických vlastnostech molekul a mají využití i v kvantitativních modelech, jako je molekulová dynamika nebo predikce pKa. Pro výpočet parciálních nábojů bylo zavedeno vícero definic postihujících elektronovou strukturu molekul různým způsobem. Výpočet nábojů přímo dle definice často vyžaduje použití výpočetně náročných ab-initio metod, a proto se využívá i méně přesných empirických metod, které na základě parametrů získaných z dříve provedených ab-initio výpočtů aproximují parciální náboje s výrazně menší výpočetní náročností. Tato bakalářská práce popisuje a srovnává dvě empirické metody z hlediska jejich schopnosti reprodukovat Mullikenovy a Merz-Kollmanovy parciální náboje, EEM (Electronegativity Equalization Method) a QEq (Charge equilibration).

Přínos práce spočívá v implementování solverů pro EEM i QEq pod svobodnou licencí, ve vytvoření ucelené databáze dříve publikovaných sad parametrů pro metodu QEq a ve vyhodnocení přesnosti obou studovaných metod a všech shromážděných sad parametrů na ověřovací sadě 657 molekul léčiv.
}

% \selectlanguage{english}
% % http://tex.stackexchange.com/questions/24066/start-new-chapter-on-same-page
% \begingroup
% \let\clearpage\relax
% \vskip 1cm plus 5mm
% \chapter*{Abstract}
% \endgroup
% 
% Partial charges, or net atomic charges, are only a crude description of the electronic structure of a molecule. Nevertheless, they are an useful characteristic of a molecule that can provide qualitative insights into physical, chemical and biological properties of a compound and can also serve as an input for quantitative computational models, be it Molecular Dynamics simulations or pKa prediction. There are several definitions of partial atomic charges that all reflect the electronic structure of the molecule in different ways and are therefore are incompatible with each other. Some definitions can be straightforwardly transformed into an ab-initio computation, which is usually impractically slow. Many empirical methods have been subsequently developed, each aims to provide results close to a particular ab-initio metod but in significantly shorter time. This thesis concerns itself with characterization and evaluation of two such empirical methods that have been developed for estimating Mulliken partial charges and Merz-Kollman partial charges, namely EEM (Electronegativity equalization method) and QEq (Charge equilibration).
% 
% This thesis improves on the previously published research by compiling a comprehensive list of previously published parameter sets for the QEq method and by providing an evaluation of the accuracy of both EEM and QEq methods on a large test set of 657 species of drug molecules. EEM and QEq solvers developed while working at this thesis are released under an open-source licence.
% \end{ThesisAbstract}

%% Thanks (optional)
%\begin{ThesisThanks}
\thesislong{thanks}{
Rád bych na tomto místě poděkoval absolventům Fakulty informatiky Stanislavu Filipčíkovi, Danielu Marekovi, Janu Pavlovičovi, Vítu Novotnému a vyučujícímu Petru Sojkovi, autorům šablony fithesis pro sazbu závěrečné práce v systému \LaTeX{}.

Při řešení bakalářské práce jsem dostal možnost provést časově náročné výpočty v počítačovém programu Gaussian 09 na počítačích Národní gridové infrastruktury MetaCentrum zpřístupněných v rámci programu "Projects of Large Infrastructure for Research, Development, and Innovations" (LM2010005).-
}
%\end{ThesisThanks}

% 
% \selectlanguage{czech}
% \begin{ThesisKeyWords}
% Huckellovy parcialní náboje, Electronegativity equalization method, Charge equilibration
% 
% \selectlanguage{english}
% \begingroup
% \let\clearpage\relax
% \vskip 1cm plus 5mm
% \chapter*{Keywords}
% \endgroup
% 
% Huckell partial charges, net atomic charges, Electronegativity equalization method, Charge equilibration
% \end{ThesisKeyWords}

%% Title page information
%\thesistitle{Srovnání semiempirických metod EEM a QEq pro výpočet nábojů v molekulách}
%\thesissubtitle{Bakalářská práce}
%\thesisstudent{Jiří Daněk}
%\thesiswoman{false} %% Important when using Slovak or Czech lang
%%\thesisfaculty{fi}  %% {fi, eco, law, sci, fsps, phil, ped, med, fss}
%\thesislang{cs}     %% {en, sk, cs}
%\thesisyear{jaro 2014}
%\thesisadvisor{RNDr. Radka Svobodová Vařeková, Ph.D.}

%% Beginning of the document
\begin{document}
%\selectlanguage{czech}

%% Front page with a logo and basic thesis information
%\FrontMatter
%\ThesisTitlePage

%% Thesis declaration (required)
%\begin{ThesisDeclaration}
%  \DeclarationText
%  \AdvisorName
%\end{ThesisDeclaration}




% 
% \MainMatter

\selectlanguage{czech}

%U diplomky se MUSÍ vejít na jednu stranu.
%\tableofcontents

%% Thesis text structured using
%% chapters, sections, subsections, etc.
\chapter*{Úvod}

Předmětem studia chemie je elektronový obal atomů, a zejména jejich valenční elektrony, neboť ty se účastní tvorby chemických vazeb. Atomové jádro se nachází hluboko uvnitř elektronového obalu a z hlediska chemických vlastností atomu hraje jen malou roli (tím, jak ovlivňuje elektronový obal). Vlastnosti molekul jsou taktéž dány jejich elektronovou konfigurací. \fixme{zdroj} V chemii existuje vícero způsobů, jak zachytit chování molekuly. Jedním z postulátů kvantové mechaniky je, že uplným popisem systému, například molekuly, je vlnová funkce \citep[s.~4]{cramer2004essentials}. Od poloviny 60.~let 20.~století se rozvíjí DFT (Density Field Theory) \citep[s.~252]{cramer2004essentials}. DFT říká, že pro určení velkého množství měřitelných údajů není nutné znát vlnovou funkci molekuly, ale snačí znát distribuční funkci elektronové hustoty. Ze znalosti vlnové funkce je možno určit distribuční funci elektronové husty. Distribuční funkci elektronové hustoty je možno určit i bez předchozí znalosti vlnové funkce, čehož se využívá při studiu velkých organických molekul, pro které je výpočet vlnové funkce zatím mimo naše technické možnosti, ale dostatečně přesný výpočet distribuce elektronové hustoty prakticky zvládnutelný je \citep[s.~260]{cramer2004essentials}. Ještě dalším zjednodušujícím popisem v porovnání s DFT jsou parcialní náboje, které aproximují elektronovou hustotu pomocí bodových nábojů umístěných v centrech atomů. Takové zjednodušení vede ke ztrátě informace, navíc neexistuje shoda na metodě, jak rozdělit elektronovou hustotu k atomům, a proto existuje mnoho různých definic parciálních nábojů. Přesto se ale jedná o velice užitečný koncept, který má řadu praktických aplikací.

Stejně jako je možné určit distribuční funkci elektronové hustoty aniž bychom před tím určili vlnovou funkci, je pomocí empirických metod možné počítat parciální náboje i bez předchozí znalosti distribuční funkce elektronové hustoty. Tato práce popisuje a srovnává dvě empirické metody pro určení parciálních nábojů molekuly na základě 3D struktury molekuly. První metodou je EEM (Electronegativiy Equalization Method) a druhou je metoda Qeq (Charge Equilibration).

Práce začíná teoretickou částí, kde jsou uvedeny a vysvětleny pojmy z oblasti počítačové chemie, se kterými jsem se při řešení práce setkal, i další poznatky, které jsem použil při řešení práce. Jsou zde také představeny obě popisované metody. Prezentuje sady parametrů pro QEq v publikované literatuře. Srovnání metod mezi sebou a s ab-initio metodou na základě které byly parametrizovány na vícero skupinách molekul. Pro různé sady molekul. 

V této práci jsem implementoval dvě empirické metody pro stanovení parcialních nábojů atomů a srovnal je z hlediska rychlosti a kvality výsledku s přístupy Ab initio. EEM a Qeq

posuzuje metody dvě empirické metody pro výpočet nábojů v molekulách EEM a Qeq z hlediska rychlosti, se soustředí na empirické metody pro rychlý výpočet parcialních nábojů atomů. Obě metody jsou nejprve uvedeny do širšího kontextu. Srovnání je provedeno na sadě "Drug targets" Welcome Trust. 

Tento problém řeší empirické metody, které se snaží kopírovat vý dávat výsledky metody podle které byly parametrizovány za výrazně menší výpočetní náročnost.

Práce je členěna následujícím způsobem. První kapitola seznamuje čtenáře se základy počítačové chemie. Ve druhé kapitole je představen problém stanovéní parcialních nábojů a metody k jeho řešení. Jsou vysvětleny principy metod EEM a Qeq. Třetí kapitola shrnuje dosavadní výsledky diplomových prací a vědeckých článků na toto téma. Čtvrtá kapitola představuje moji implementaci obou metod a . Pátá kapitola obsahuje srovnání metod EEM a Qeq na testovací sadě molekul. Šestá kapitola shrnuje dosažené výsledky.

%This thesis is structured as follows. The problem of partial atomic charge estimation is first introduced assuming no computational chemistry experience. The principle of EEM and Qeq is explained. Both methods are implemented, parameter sets from previous published literature are compilled. Methods are evaluated by comparing the predicted charges with a gold standard of computed using the ab-initio method on Test set of drug published by the Welcome Trust regarding \todo{accuracy} using several statistical measures. The merits of this thesis of the published QEq parameter set and in evaluation of both methods with all parameter sets on a large set of ??? inorganic molecules.

\chapter{Základní pojmy počítačové chemie}

Předmětem počítačové chemie (chemoinformatiky, někdy také cheminformatiky) je využití počítačů v chemii. To obnáší jak digitální archivaci výsledků chemického výzkumu a následné vyhledávání v těchto datech, tak provádění výpočtů a simulací chemických a biochemických systémů s cílem formulovat nové hypotézy nebo zamítnout hypotézy stávající. Příkladem z první oblasti jsou databáze chemických sloučenin, reakcí, a faktografických údajů, které začaly vznikat od 60. let 20. století v reakci na v té době začínající exponencialní nárůst množství chemických poznatků. Do té doby narůstal počet známých chemických sloučenin přibližně lineárně. Když se tento trend změnil v exponenciální, chemici velmi rychle spatřili potenciál výpočetní techniky umožnit jim se s touto informační explozí vypořádat. Příkladem z druhé, mladší oblasti počítačové chemie, která se nazývá výpočetní chemie (computational chemistry) jsou metody molekulové mechaniky a molekulové dynamiky, které umožňují studovat chování molekul in-silico. Definic výpočetní chemie existuje mnoho, autorem následující je Schlyer. Přeložena z angličtiny zní: \uv{Počítačová chemie usiluje o to, popsat všechny aspekty chemie v co nejlepší shodě s realitou pomocí výpočtů namísto experimentu.} \citep{ma1995philosophy}.

Chemoinformatika může být použita k předpovídání vlastností molekul, její předpovědi a simulace je ale nutno nakonec vždy ověřit experimentálně. Přínos chemoinformatiky spočívá v tom, že dovoluje výzkumníkům zaměřit se na perspektivní experimenty, když před tím rychle a levně zamítne ty neperspektivní. Konkrétním příkladem je virtualní screening, kdy se pomocí výpočetních metod vybírají kandidátské molekuly z rozsáhlé knihovny a zamítají se takové, které například mají nevhodné pKa.

\section{Reprezentace molekuly v počítači}

Předmětem zájmu počítačové chemie je molekula. V počítači můžeme molekulu reprezentovat buď z hlediska její struktury, nebo geometrie. \citep{s.~2}{leach2001molecular}

\subsection{Strukturní reprezentace}

Molekula sestává z atomů, mezi kterými jsou chemické vazby. Strukturní reprezentace molekuly je vrcholově ohodnocený pseudograf. Pseudograf je multigraf, který může obsahovat smyčky (hrany vycházející i vcházející do stejného vrcholu). Multigraf je graf, mezi jehož dvěma vrcholy může být i více než jedna hrana (a na jednom vrcholu může být i více než 1 smyčka). Vrcholy v pseudografu představují atomy v molekule, hrany v grafu představují chemické vazby mezi atomy, vrcholové ohodnocení grafu nám umožňuje přiřadit vrcholům chemické symboly prvků. Možnost vložit mezi dvojici vrcholů více než jednu hranu se využije pro popis vícenásobných vazeb. Možnost přidávat smyčky slouží k zachycení volných elektronových párů na atomech.

Formálně je struktura molekula popsána pseudografem $G = (V, E, L, j, b)$, kde $V$ je množina vrcholů, $E$ je multimnožina hran, $L$ multimnožina smyček a $j$ je funkce, přiřazující prvku množiny $V$ ohodnocení z množiny $b$.

Grafová reprezentace molekuly odpovídá strukturnímu vzorci, běžně používanému zejména v organické chemii. Využití grafů k reprezentaci molekul otevírá prostor využití grafových algoritmů k řešení chemických problémů. Tím nejzákladnějším, abychom byly schopni s molekulami v počítači pracovat, je jejich ekvivalence. Zjistit, zda dva grafy reprezentují stejnou molekulu znamená řešit problém izomorfismu grafů. Pro obecný graf je tento problém NP-úplný \citep{s.~335}{skiena1998algorithm}. Izomorfismus planárních grafů je možno rozhodnout v čase O(n log n). Při řešení izomorfizmů molekulových grafů je možné využívat heuristiky, které se snaží co nejrychleji najít nesoulad mezi porovnávanými molekulami a kandidáta zamítnout, čímž dokáží dramaticky snížit časovou složitost průmětného případu. Molekulové databáze také mohou předpočítat indexy zrychlující následné dotazy.

\subsection{Geometrická reprezentace}

Zápis molekuly pomocí její geometrií zanedbává informaci o vazbách a popisuje pouze pozice atomů v prostoru. Pozice atomů mohou být udány buď v kartézských souřadnících vůči libovolně zvolenému počátku souřadnic, nebo ve vnitřních souřadnicích, pomocí vzdáleností a úhlů vůči dříve uvedeným atomům.

Zadání v kartézské soustavě souřadnic znamená o každém atomu uvést jeho typ a trojici souřadnic x, y, z.

Zápis v interních souřadnicích se provádí pomocí Matice Z. U prvního atomu je uveden pouze jeho typ. Druhý atom je zadán typem a vzdáleností od prvního. Ke třetímu atomu je uvedena kromě typu ještě vzdálenost od druhého atomu a úhel mezi prvním, druhým a třetím atomem. Čtvrtý atom (a každý další) je určen typem, vzdáleností od předchozího atomu, úhlem mezi ním a předchozími dvěma a torzním úhlem rovin, kde první rovina je určených předchozími třemi atomy a druhou rovinu zadává poslední atom spolu s předchozími dvěma. Pro popsání obecné $N$-atomové molekuly potřebujeme $3N-6$ souřadnic. Číslice 6 se vztahuje k šesti nespecifikovaným stupňum volnosti: translaci a rotaci celé molekuly v trojrozměrném prostoru.

Geomertii molekuly je možno určit výpočetně pomocí optimalizace energie molekuly do základního stavu (globálního minima). Energii molekuly je možno vypočítat buď kvantově mechanickými molekulami, nebo molekulovou mechanikou. Empiricky je možné odhadnout geometrii molekuly pomocí porovnávání s jinými molekulami, pro něž je geometrie již známa. Experimentální určení struktury molekuly je možno provést pomocí Roentgenové krystalografie, nebo spektroskopickými metodami.

K základním algoritmům pro práci s geometrickou reprezentací molekuly patří převod mezi interními a kartézskými souřadnicemi, nebo přikládání molekulárních geometrií k sobě tak, aby mezi nimi byla co největší shoda. Počítačová grafika nachází aplikace při vizualizaci molekulových geometrií. Tyto vizualizace dnes z velké části nahradily dříve používané fyzické modely molekul.

\subsection{1D, 2D a 3D reprezentace molekuly v databázích}

1D daty se rozumí bibliografické informace o molekule, 2D daty se rozumí strukturní grafová reprezentace molekuly a 3D daty geometrická reprezentace molekuly.

\subsection{Souborové formáty}

Souborových formátů existuje v počítačové chemii velmi mnoho, protože autoři softwaru často definovali vlastní formát pro každý software. Z vlastní zkušenosti jsem nabyl dojmu, že nejpoužívanější formáty jsou SDF, PDB, XYZ a SMILES.

Formít PDB (Protein Data Bank) http://www.wwpdb.org/documentation/file-format MOL/SDF (Structure-Data File) uchovává molekuly v strukturní? reprezentaci. http://cactus.nci.nih.gov/SDF_toolkit/ Molecular Design Limited. SDF je rozšíření MOL které umožňuje přidat k molekule attributy. XYZ je vpodstatě výčet prvků společně s jejich kartézskými souřadnicemi v molekule. http://openbabel.org/wiki/XYZ_%28format%29

SMILES (SImplified Molecular Input Line Entry Specification) je strukturní formát zajímavý tím, že zápis molekuly má podobu textového řetězce. Nad formátem SMILES je postaven dotazovací jazyk, který je pro molekulární struktury jako regulární výrazy pro textové řetězce.

Pro práci s molekulovými formáty je často používaný OpenBabel. Jedná se o softwarovou knihovna a sadu utilit pro převod mezi velkým množstvím používaných formátů.

\section{Ab initio, semiempirické a empirické metody}

Latinská fráze \emph{ab initio}, která se překládá jako „z prvotních principů“, v kontextu výpočetní chemie označuje výpočetní metody, které vycházejí z kvantové mechaniky. Jedním z postulátů kvantové mechaniky je, že vlnová funkce je úplnou charakteristikou systému. Ab initio metody fungují na principu určení vlnové funkce a z ní následně žádaných veličin. Někteří autoři do této skupiny zahrnují i metody založené na funkcionálu elektronové hustoty (DFT,  \textit{\foreignlanguage{english}{Density Functional Theory}}), jiní je vyčleňují jako samostatnou kategorii.

DFT metody jsou co se týče principu výpočtu velmi podobné ab initio metodám. Namísto vlnové funkce se pro charakteristiku systému používá funkcionál elektronové hustoty, který sice není úplnou charakteristikou systému, pro mnohé aplikace ale plně dostačuje, a navíc je jeho výpočet realizovatelný i pro velké biomolekuly, pro které by určení plné vlnové funkce trvalo příliš dlouho.

Semiempirické metody jsou modifikací ab initio metod takovou, že tam, kde to výrazně přispěje ke zrychlení výpočtu, se za cenu nižší přesnosti použije předem určených parametrů, které mohou být zjištěny i experimentálně.

Empirické metody staví na jiných teoretických základech, než je kvantová mechanika. Příkladem může být molekulová mechanika, technika pro určování konformace a dalších parametrů molekul ze zjednodušeného modelu molekuly, který je založený na zákonech klasické fyziky. Využívání experimentálních parametrů je u empirických metod téměř pravidlem.

%Jedno z kritérií kvality semiempirických a empirických metod se nazývá přenositelnost parametrů. V praxi se ukazuje, že v rámci jednotlivých tříd strukturně podobných molekul bývají empirické parametry pro jednotlivé molekuly velmi podobné, a proto je možné stejnou parametrizaci použít pro výpočty nad všemi molekulami dané třídy. Čím je míra přenositelnosti parametrů vyšší, neboli čím širší škálu molekul můžeme dostatečně přesně popsat s použitím jedné parametrizace, tím lépe vyhodnocovaný model generalizuje realitu.

%Klasickou Newtonovskou mechaniku je možno označit za empirickou teorii. Gravitační konstanta je empirický parametr. Relativistická fyzika: tuto konstantu vypočítat.

%\subsection{Hyperplocha potencialní energie}
%
%Born-Oppenheimerova aproximace, která předpokládá fixní pozice jader dala vzniknout konceptu hyperplochy potencialní energie (PES, Potential Energy Surface). Hyperplocha potencialní energie je množina bodů v $1+3N-6$rozměrném prostoru, kde posledních $3N-6$ souřadnic představuje pozice jader atomů a první souřadnice je energie základního stavu molekuly s jádry na těchto pozicích.

\subsection{Molekulová mechanika}

Molekulová mechanika popisuje chování molekul pomocí principů z klasické fyziky. Modely, kterými se atomy řídí, se nazývají silová pole. Atomy jsou v molekulové mechanice chápány jako jako hmotné body, vazby mezi nimi se chovají podle Hookova zákona pro pružiny, rotace kolem vazby je modelována obvykle pomocí harmonického potenciálu atd. Při návrhu silových polí se klade důraz na co nejmenší výpočetní náročnost. To je důvod proč se energie vazby nepopisuje Moorovým potenciálem, ale kvadratickým polynomem, který je aproximací Moorova potenciálu v okolí rovnovážné délky vazby. Molekulová mechanika ve své základní formě neumí popsat vznik a zánik chemických vazeb. Na druhou stranu je výpočetně výrazně méně náročná, než kvantová mechanika. Typické silové pole UFF. Protože molekulová mechanika modeluje vazby mezi atomy, pracuje se strukturním zápisem molekuly.

\section{Kvantová mechanika}

Kvantová mechanika pracuje s geometrickým zápisem molekuly.


Počátek kvantové fyzkyky se vztahuje k výsledkům Maxe Planka při výzkumu záření černého tělesa a Alberta Einsteina fotoelektrický jev kteří při popisu těchto jevů vyšli z předpokladu, že energie světla je kvantována na násobky Plankovy konstanty.. Ve věci záření černého tělesa v té době existovaly dva hlavní výsledky.  ale “ultrafialová katastrofa”. Druhá rovnice popisovala dobře chování v ultrafialové oblasti, ale selhávala při popisu nižších energií. Přínos M. Planka spočívá v tom, že nejprve nalezl rovnici, ve které spojil chování obou hypotéz tak, aby výsledná rovnice dávala správné výsledky pro všechny frekvence. Dále potom nalezl fyzikální vysvětlení pro takto vzniklou rovnici. 

Vlnově částicová dualita. Louis de Broglie přišel s hypotézou, kdy všem objektům přiřadil de broglieho vlnu a uvedl vzorec k výpočtu její vlnové délky. (http://en.wikipedia.org/wiki/Matter_wave, najít lepší zdroj) V roce 1927 byla pozorována difrakce elektronů a pozorovaný difrakční obrazec přesně odpovídal frekvenci předpovězené de broighliho rovnicí.

Na výsledky De Broglieho navázal Ervin Schrodinger, který v roce 19xx nalezl diferencialní rovnici, ze které je možno vypočíst funkci, popísující amplitudu de broigliho vlny ve všech bodech prostoru.

Kvantová mechanika se postupně vyvýjela 
kvantová elektrodynamika

Hamiltonova idea charakteristické funkce, kterou poprve použil ve svém díle věnovaném paprskové optice. Každé optické soustavě přiřadil charakteristickou funkci, která popisovala veškeré studované vlastnosti soustavy. Operátor celkové energie se proto nazývá Hamiltonián.


Základní rovnice kvantové fyziky ve tvaru

\[E \psi = e \psi\]

kde E je operátor, e je vlastní hodnota operátoru a \psi je vlnová funkce systému.

Vlnová funkce je funkcí 
\[\psi : R^3 -> C
(x, y, z) -> a\]

(FIXME jednotky?)

která souřadnicím v prostoru přiřadí amplitudu v tom místě. Tato amplituda může být i komplexní číslo a nemá fyzikální význam. Absolutní hodnota z druhé mocniny vlnové funkce v určitém místě udává hustotu pravděpodobnosti, tedy pravděpodobnost, že se elektron vyskytuje v malém okolí bodu x, y z.  (FIXME co pro víceelektronvou vlnovou funkci, pst čeho?)
Jedna z možností, jak o vlnové funkci uvažovat, je představit si ji jako neprůhledné orakulum, které nám umí vyjevit určitou fyzikální charakteristiku systému, pokud je tázáno vhodným operátorem.

\[H \psi = E \psi\]

Hamiltonián vypadá následovně

Tuto nehezkou rovnici je možno vylepšit tím, že použijeme atomové jednotky. Jedná se o alternativní měrnou soustavu, která je zvolena tak, aby konstanty v rovnicích v atomových jednotkách byly rovny jedné a nebylo nutno je psát.

Operátor celkové energie je možno rozdělit na operátory kinetické a potencialní energie

kinetická energie elektronů, repulze elektronů, repulze jader, přitahování elektronu a jader

Schrodingerova rovnice je analyticky řešitelná pouze pro systémy jedna částice v potenciálovém poli. To odpovídá vodíkovému atomu, případně kationtu helia?

Aproximace
Z tohoto důvodu je nutno přistoupit k následujicím aproximacím.

Born Oppenheimerova aproximace
Při výpočtu budeme uvažovat atomová jádra jako nepohyblivá. Souřadnice jader tedy nebudou vystupovat jako proměnné, ale jako konstanty. Tato idea je naprosto klíčová pro umožnění řešení schrodingerovy rovnice a z fyzikálního hlediska je opodstatněná, neboť  v kontrastu s elektrony se 18xxkrát hmotnější jádra opravdu dají považovat za stacionární. (Výpočetní chemie) Dalším přínosem této představy je idea hyperplochy potencialní energie (PES potential energy surface).

Nevím jaká
Při řešení schrodingerovy rovnice pro víceelektronový atom zanedbáme člen vzájemné interakce elektronů. V praxi se ukazuje, že tento krok vede k tomu, že energie získaná výpočtem je asi o xxx vyší, než kdybychom interakce elektronů uvažovali. Tato energie se nazývá korelační energie. Představa za tímto rozdílem je taková, že záporně nabyté elektrony se navzájem odpuzují. Když tyto interakce zanedbáme, xxx okolo jádra pohybují takovým způsobem


Orbitalová aproximace
Při řešení schrodingerovy rovnice pro víceelektronový atom můžeme začít počítat 
Řešením schrodingerovy rovnice pro vodík dospějeme k celé řadě řešení, která splňují fyzikální podmínky. Pokud bychom si nechali vykreslit oblasti v prostoru, kde je  

(výpočetní chemie) Schrodingerova rovnice má, jako každá jiná diferencialní rovnice, nekonečný počet řešení. Kvantová chemie klade na vlnovou funkci následujicí požadavky: vlnová funkce musí být , a tedy spojitá, integrál druhé mocniny přes celý (třídimenzionalní) prostor se musí rovnat jedné. Dále je nutno zahrnout Pauliho princip výlučnosti.

Obecným přírodním principem je, že systémy se snaží zaujmout konfiguraci, které příšluší minimální potencialní energie. Řešení schrodingerovy rovnice spočívá ve snaze najít takovou vlnovou funkci systému, která mu přiřazuje nejmenší energii. 

Teorie elektronové hustoty (Počítačová chemie)
Bylo odvozeno, že celková energie soustavy (vlastní hodnota hamiltoniánu vlnové funkce) je určena ze znalosti funkce popisující eletronovou hustotu. Teorie funkcionálu elektronové hustoty tohoto faktu využívá pro

(FIXME co je tady vstupem a výstupem a co se z toho dá všechno odvodit)

Ab initio
Výpočty pomocí schrodingerovy rovnice a Born oppenheimerovy aproximace mají asymptotickou složitost O(n^4), výpčty na základě teorie elektronové hustoty potom O(n^3). Kde n je počet elektronů v molekule. Protože pro malá n se naopak výpočet na základě teorie elektronové hustory ukazuje řádově pomalejsí. A pro velká n jsou obě metody neprakticky pomalé.

Výhodou funkcionálu elektronové hustoty je, že výpočet je možno relativně jednoduše provést semiempiricky. Tedy za pomoci vhodne parametrizace výpočet zkrátit. Ukazuje se, že kvantově mechanické výpočty v chemii je funkcionál elektronové hustoty vždy efektivnější.

Při používání této metody je nutno dávat si pozor na následující dvě fakta. Výpočet optimalizuje elektronovou hustotu, nikoli celkovou energii. (FIXME nechápu)

Jak již bylo řečeno, vlnová funce jednoznačně určuje všechny parametry kvantového systému. To ovšem neplatí o elektronové hustotě. Zatímco z vlnové funce je možno vypočítat elektronovou hustotu, naopak to možné není. Proto pokud se veličiny, které se z výpočtu cheme dovědět,  nedají vypočítat z funkcionálu elektronové hustoty, nezbývá než řešit schrodingerovu rovnici.




Základním pojmem kvantové mechaniky je vlnová funkce $\Psi$, která je úplným popisem systému, v našem případě molekuly. Hodnoty pozorovatelných veličin z vlnové funkce získáme tak, že na vlnovou funkci aplikujeme operátor, který veličině přísluší. Operátorem celkové energie systému je Hamiltonián $H$ a jeho aplikací na vlnovou funkci dostaneme Schrodingerovu rovnici, která ve své na čase nezávislé podobě zní $H\Psi = E\Psi$.

Operátory jsou funktory, tedy funkce, které mají jako argument funkci, a jejich hodnotou je také funkce. Pro osvětlení tohoto pojmu je možné uvést operátor derivace. Operátorová rovnice s operátorem derivace může vypadat $\frac{d}{dx}f = af$. Řešení této rovnice je například funkce $e^x$ a obecně funkce ve tvaru $e^x$.

Přípustné stavy systému jsou vlastní funkce Schrodingerovy rovnice. Vlastní funkce s nejmenší energií odpovídá základnímu stavu molekuly.\Psi operátorové rovnice. Řešit Schrodingerovy rovnice znamená určit takovou vlnovou funkci, která vyhovuje operátorové rovnici a přísluší jí nejmenší energie.

 \fixme{haček nad H}

kde H je operátor celkové energie, a E je energie příslušného stavu. Základní stav systému je stav s nejmenší energií.

Schrodingerovu rovnici je možné analyticky řešit jen pro chemické systémy s jedním elektronem, to jest atom H, kationty He+, Li2+ a tak dále. Ve složitějších případech je nutno hledat řešení numericky. Analytické řešení vodíkového atomu je významné tím, že jednoelektronové vlnové funkce se nazývají orbitaly a ve schématu LCAO (Linear Combination of Atomic Orbitals) se používají k vyjádření složitějších víceelektronových funkcí. Nabízí se srovnání s harmonickou analýzou, která signály reprezentuje jako linearní kombinaci harmonických složek. Stejně jako v případě harmonické analýzý se jedná pouze o matematickou reprezentaci reality, která nám má poskytnout vhled do problému, nikoli o fyzikální popis reality; nikdo se určitě nedomnívá, že například hudební nástroje obsahují spoustu malých harmonických oscilátorů, které zní dohromady a tvoří výsledný zvuk.

levels of theory

Kvantově mechanické metody se zapisují značkou metoda/báze. Metody jsou například Hartree-Fock (HF), PM3 (Parametrized Model 3). Báze jsou.

Bohr Oppenheimerova approximace, která

elektrony a jádra jsou bodové náboje
pozice jader je fixní a předem daná
elektrony jsou nezávislé, se navzájem neovlivňují

Druhý z předpokladů je oprávněný proto, že jádra jsou řádově hmotnější než elektrony a pohybují se mnohem pomaleji. Jádra atomů ve skutečných molekulách vybrují okolo svých rovnovážných poloh. Můžeme tedy považovat pozice jader za fixní a 

Časté výpočty v kvantové chemii jsou. Optimalizace struktury molekuly do stavu s nejnižší energií. Pro výpočet energie se použije kvantový výpočet.

Harree Fockova metoda

řeší se optimalizační problém nalezení takové \(\Psi\), vyjádřené jako linearní kombinace bázových orbitalů, aby E byla minimální.

Jako bázové orbitaly se používají jednoelektronové orbitaly získané analytickým řešením vodíkového atomu.

HF/3

metoda/báze

Limity aproximace

Korelační energie elektronů je značná \citep{s.~132}{cramer2004essentials}, z čehož plyne, že Born Oppenheimerova aproximace o nezávislosti elektronů je chybná. Naštěstí je možné těžit z toho, že chyba způsobená zanedbíním elektronové kolelace je u podobných molekul podobně velká, a tak to pro mnoho praktických aplikací nevadí. Pokud je potřeba zahrnout do výpočtu elektronovou korelaci, je potřeba použít metodu, která to umožňuje, má to ale velmi nepříznívý dopad na dobu výpočtu.

Volba báze má vliv na výsledek výpočtu. Čím je báze větší, tím lépe je systém popsán a tím nižší bude vypočtená energie, a to až do dosažení Hatree-Fockovy limity. Orbitaly jsou umístěny na jádrech atomů. Pokud se ve výpočtu nachází více atomů vedle sebe, bude v okolním prostoru přítomno mnoho orbitalů. Pokud chceme srovnávat energie reaktantů a produktu reakce, musíme si uvědomit, že v molekule reaktantu je přítomno mnohem více orbitalů. Řešením je přidat virtuální atomy rektantu A do reaktantu B a naopak. Basis Set Superposition Error




empirické metody, které 

zanedbává korelační energii

program gaussian, leach s. 8


\chapter{Parcialní náboje}

Parcialní náboje jsou velmi starý chemický koncept, který popisuje distribuci elektronové hustoty v molekule pomocí bodových nábojů lokalizovaných na jádrech atomů. Parcialní náboje dovolují formulovat kvalitativní souvislosti některé vlastnosti molekuly jako celku a topologii jejích chemických vazeb. (dipólové momenty, ...) 

Pro jednotlivé volné atomy definujeme jejich elektronegativitu, realné číslo vyjadrující jejich schopnost přitahovat elektrony. Existuje vícero definic elektronegativity (Paulingova, experimentální měření, ...)  \todo{Naopak valenční elektrony .}

Mezi parcialním nábojem a elektronegativitou existuje přímá souvislost. Samostatné atomy jsou elektricky neutální. Při utváření molekulových vazeb atomy navázané na ty s vyšší elektronegativitou nabývají parcialní kladný náboj v důsledku ztráty valenčních elektronů ve prospěch svého vazebného partnera, zatímco atomy s méně elektronegativním vazebným partnerem od něj elektrony přebírají. \todo{toho Sandersona můžu dát už sem}

Pomocí bodových nábojů na atomech není možné přesně popsat distribuci elektronové hustoty v molekule. Proto se při definování schématu výpočtu parciálních nábojů musejí autoři vždy zaměřit na určitá kritéria. Například Hückelovy náboje jsou výpočetně nenáročné (pokud uvažujeme jen samotnou populační analýzu a neuvažujeme kvantově mechanické vypočty, které jí předcházejí) a dají se použít k předpovědím reaktivity molekul, na druhou stranu nedokáží předpovědět fyzikální vlastnosti molekuly. Náboje dobře postihují elektrický potenciál okolo molekuly, u velkých organických molekul ale přiřazují nesmyslně velké náboje vnitřním atomům molekuly, a proto se nehodí na molekulově mechanické simulace.

Metody založené na ekvalizaci náboje  

Na ekvalizaci náboje vznikla celá řada metod, 1985 EEM, 1991 QEq, SQE pro rychlý výpočet náboje, od každé z těchto metod dále existují odvozené metody. Hodnocení korelace. Parametrizují se Mullikenovy, DFT, nějaké další (ty benchmarky), úspěšnost se hodnotí std odchylkou, korelací. Parametrizují se i parametry molekul, dipolové momenty, 

Jednou z metod stanovéní parcialních nábojů biomolekul je a fitování podle experimentu. Metoda . Okolo molekuly se vytvoří myšlená mřižka a metodou nejmenších čtverců se stanoví náboje na atomech tak, aby reprodukovaly experimentálně naměřené hodnoty v bodech této mřížky.

mřížka je citlivá na rozmístění bodů? že se naučí mřížku a ne molekulu, resp distribuci náboje kolem molekuly, přeučení? nebo že různá mřížka, různé výsledky?

Náboje na atomech uvnitř molekuly nejsou významné a metoda má tendence stanovit je nerealisticky obrovské.

R  Restricted přidává další omezení aby se předešlo výše uvedeným problémům.

\section{}

kuprov2012lecture5a

\subsection{Mullikenova populační analýza}

Mulikenova populační analýza přiřadí atomu $A$ parcialní náboj

\[Q_A = - \sum_{\mu \in A} \sum_{v \in A} P_{\mu v} S_{\mu v} - \frac{1}{2} \sum_{\mu \in A} \sum_{v \in A} P_{\mu v} S_{\mu v}\].

PS jsou překryvové integrály

Právě uvedený vzorec znamená, že se provede sumace přes všechny dvojice bázových funkcí $\mu$ a $v$. Pokud se obě funkce nacházejí na stejném atomu, přiřadí se celá příslušná hustota tomuto atomu. Pokud se funkce nacházejí na různých atomech, připočte se polovina hustoty prvnímu atomu a polovina druhému atomu.

Výhodou Mullikenovy populační analýzy je jednoduchost jejího výpočtu (z vlnové funkce) a fakt, že určené náboje můžeme s úspěchem použít pro předpovídání reaktivity molekul. Nevýhodou je fakt, že elektronová hustota v difúzních orbitalech je započtena jen k atomu, na kterém se orbital nachází, a také rovnoměrné rozdělení elektronové hustoty mezi oba dva atomy, což je zcela arbitrární pravidlo, které je navíc v rozporu s chemickou intuicí. 

\subsection{Další schémata}

Löwdinova populační analýza rozděluje elektronovou hustotu mezi atomy nerovnoměrně.



%An alternative measure of the charge distribution involves a partitioning into partial atomic charges. While such partitioning is always arbitrary (see Chapter 9) simple methods tend to

%Valence orbitals, on the other hand, can vary widely as a function of chemical bonding. Atoms bonded to significantly more electronegative elements take on partial positive charge from loss of valence electrons, and thus their remaining density is distributed more compactly. The reverse is true when the bonding is to a more electropositive element. From a chemical

%Highly charged regions of molecule are the most reactive locations
%Charges = clue to reactivity
%Charges provide a very useful information about a molecule
%Charges provide a deep insight into a chemical behaviour of a molecule


\todo{vyuziti naboju k fitovani vlastnosti}
%Charges are excellent descriptors in QSPR and are necessary in simulations = molekulová mechanika
%predikce fyzických, chemických a biologických vlastnosti
% a jsou parametrem vstupem mnoha výpočetních modelů na předpovídání

%výpočet stability molekul

%odhad směru průběhu chemické reakce

%předpovědi interakcí s dalšími molekulami

\todo{parcialní náboje mají fyzikální smysl}

\todo{experimentalni dukazy spektroskopie}

\todo{polarita molekul}

\todo{dipolove momenty, ...}

%značně nepřesná charakterizace

%Parametrizace EEM pro biomolekuly

%Roentgenová krystalografie

Rozložení elektronové hustoty v molekule je pozorovatelná veličina. Existuje pro ně kvantový operátor a možno pozorovat například pomocí Roentgenové krystalografie. je možno měřit, přiřazení elektronů k atomům je pouze zjednodušující popis realné distribuce elektronové hustoty. Elektronová hustota je rozložena mezi atomy, nikoli na atomech a neexistuje experimentální metoda pro určení nábojů na atomech.

\chapter{Parcialní náboje}

Ve stejnojaderných dvouatomových molekulách působí obje atomová jádra na elektrony stejně a elektronová hustota (vazebných elektronů) je taktéž rovnoměrně rozdělena mezi obje jádra. Vazby, podél kterých je elektronová hustota rozdělena rovnoměrně se nazývají nepolární. Dvouatomové molekuly složené z různých atomů mají elektronovou hustoru rozloženu podél vazby nesymetricky, protože jeden z atomů (v důsledku rozdílného protonového čísla a elektronové konfigurace) přitahuje elektrony větší silou, než ten druhý. Takové vazby se nazývájí polární, nebo, pokud jsou elektrony velmi silně přitahovány k jenomu z atomů, iontové. Ve víceatomových molekulách dochází k přesunu elektronů přes více než jednu vazbu. Vysoce elektronegativní substituenty v molekulách derivátů uhlovodíků (například halogeny) snižují elektronovou hustotu na navázaném uhlíkovém skeletu. S rostoucí vzdáleností od substituovaného atomu se účinek tohoto efektu snižuje. \citep[s.~120]{benešová2002odmaturuj}

Elektronová hustota se nachází mezi atomy tvořícími vazebný pár, nikoli na atomech. Přiřazení elektronové hustoty k atomům, ač technicky ne zcela přesné, se ukázalo být velmi praktické, jak už při zápisu vzorců, tak při výpočtech.

Experimentálně měřitelným projevem nerovnoměrného rozložení elektronové hustoty v molekule je dipólový moment molekuly. Permanentní dipól molekuly se projeví ve spektru získaném mikrovlnnou spektroskopií. Příkladem z praktického života je ohřev vody v mikrovlné troubě, kdy magnetické pole trouby způsobí rotaci polárních molekul vody. Dipól dvouatomové molekuly s parcialními náboji $-q$ a $q$ ve vzálenosti $r$ je $\mu = qr$. Dipól obecné molekuly je vektorový součet $\mu = \sum_j q_j p_j$, kde $q_j$ je náboj a $p_j$ je vektor souřadnic $j$-té molekuly.

Schopnost atomu v molekule přitahovat elektrony zachycuje veličina zvaná atomová elektronegativita, kterou zavedl Linus Pauling a bývá značená obvykle $X$. Metod pro její výpočet existuje několik, například $X = k(I+A)$, kde $I$ je ionizační energie a $A$ je elektronová afinita příslušného atomu. Autorem právě uvedené definice je R. S. Mulliken. Elektronegativita je relativní veličina a je vztažena ke zvolenému referenčnímu prvku, kterým bývá zpravidla fluor. prehledChemie s 104 atkins 380

Parcialní náboje jsou reálná čísla, která popisují podíl elektronové hustoty příslušející k jednotlivým atomům v molekule. Z této definice vyplývá, že parcialní náboje jsou experimentálně neměřitelný teoretický koncept, protože ačkoli elektronovou hustotu je možno přesně změřit například pomocí rentgenové krystalografie nebo ji vypočíst pomocí ab initio metod, schéma jejího přiřazování k atomům musí být nutně arbitrární. Pro výpočet parcialních nábojů bylo zavedeno mnoho metod, z nichž každá přiřazuje elektronovou hustotu k atomům jiným způsobem, a proto také dávají navzájem rozdálné číselné výsledky.

V chemické teorii se běžně nepracuje s absolutními hodnotami parcialních nábojů, ale s jejich rozdíly a trendy. \fixme{zdroj?} Pomocí rozdíů v hodnotách parcialních nábojů se například v organické chemii vysvětlují jevy jako vodíkové vazby nebo reaktivita funkčních skupin (pojmy jako indukční a mezomerní efekt). \fixme{zdroj: přehled sš chemie, vodíkové vazby i atkins} Význam parcialních nábojů v chemii je tak především kvalitativní.

Parcialní náboje je možno používat i kvantitativně, například jako jeden ze vstupů regresního modelu k predikci disociačních konstant. V tom případě je ale vždy dáno, jakou metodou je nutno náboje počítat.

%Problém je s přiřazením náboje k jednotlivým atomům. K tomu existují různé metody, které dávají různé výsledky.

%Příčinou vzniku atomových nábojů je nerovnoměrné rozložení elektronů v molekule.

%, úzce souvisí s rozložením elektronové hustoty v molekule poskytují celou řadu užitečných informací. Simulace

Kritériem správnosti kvantitativních výstupů jsou užitečnost a obecnost výpočetní metody. Užitečností se myslí existence experimentu, s nímž výpočetní metoda vykazuje dobrou, jinými slovy, který dobře modeluje. Obecnost je parametrem u empirických metod a jedná se primárně o míru přenositelnosti parametrů.

\section{Metody výpočtu}

Metody pro výpočet nábojů se dají rozdělit do čtyř kategorií. \citep[s.~50]{cramer2004essentials}

Třída I představuje empirické metody, které vycházejí nikoli z kvantové mechaniky, ale jsou založeny na fyzikálních analogiích a intuici tvůrců. Tyto metody mohou využívat experimentálních dat, jako jsou dipólové momenty nebo elektronegativity.

\todo{definice elektonové hustoty a vzorec?}

Třídy II a III zahrnují metody, které vycházejí buď přímo z vlnové funkce (třída II), nebo z pozorovatelné veličiny z vlnové funkce vypočtené (třída III), například z elektronové hustoty, a na základě intuitivně zvoleného schématu ji rozdělí na příspěvky od jednotlivých atomů.

Třída IV je vyhrazena metodám, které vycházejí z metod ve třídách II a III, rozdělení nábojů ale hledají takové, které nejlépe odpovídá experimentálně určeným parametrům, například dipólovému momentu molekuly.

V praxi se nejčastěji používají hodnoty parcialních nábojů získané metodami ve třídách II a III. V případě, že pro požadovanou aplikaci je výpočet některou z metod tříd II nebo III příliš časově náročný, je možné náboje aproximovat použitím vhodné metody z třídy I.

Ekvalizace elektronegativity

Elektronegatita je realné číslo obvykle v rozsahu \todo{rozsah} Je možno chápat jako schopnost přitahovat elektrony. Při pohledu na periodickou tabulku platí, že při pohybu směrem doprava a nahoru elektonegativita roste. Existuje mnoho schémat pro výpočet elektronegativity. Elektronegativita se definuje pro volné atomy.

Vznik molekuly spočívá ve sdílení elektronů mezi vícero atomy. Ustavení molekuly můžeme chápat jako ustavení rovnovážného stavu v uspořádání elektronového oblaku. Pokud by se elektonegativita zvoleného atomu zvýšíla, bude více obklopen elektrony, což jeho elektronegativitu opět sníží.

Metody equilibrace náboje jsou motivovány vyjádřením potencialní energie atomu v závislosti na náboji a minimalizováním potencialní energie za podmínky známé hodnoty celkového náboje molekuly.

Elektronegativitu můžeme chápat ionizačního potencálu a elektronové afinity izolovaných atomů, nebo jako další parametry.

\section{Electronegativity Equalization Method (EEM)}

V článku Electronegativity Equalization: Application and Parametrization



Metoda Electronegativity equalization

tří předpokladů

efektivní elektronegativita atomu v molekule je elektronegativita izolovaného atomu, snížená nebo zvýšená o

elektrony v molekule se přeskupí od méně elektronegativních atomů k více elektronegativním atomům tak, že efektivní elektronegativita všech atomů v molekule bude stejná.

Pro každý atom můžeme napsat rovnici ve tvaru

Všechny tři rovnice můžeme úsporně vyjádřit maticovým zápisem ve tvaru


EEM je první posuzovaná empirická metoda pro výpočet parcialních nábojů.

vychází ze tří podmínek. První podmínkou . Druhá podmínka, která dala metodě její jméno, je Sandersonův princip ekvalizace elektronegativity, a třetí podmínkou je zachování celkového náboje molekuly. První podmínka popisuje závislost potencialní energie molekuly na parcialních nábojích, na vzdálenosti atomů, jejich elektronegativitě \todo{a}. Za předpokladu, že studovaná molekula je v rovnovážném stavu můžeme usoudit, že se nachází ve stavu s nejnižší potencialní energií. Víme tedy, že deriace výrazu pro energii musí být nulová. Tímto krokem jsme odstranili energii jako neznámou a z tohoto vztahu si můžeme vyjádřit efektivní elektronegativitu každého z N atomů v molekule jako funkci jeho náboje. Druhá podmínka nám umožňuje vytvořit N-1 rovnic porovnávajících elektronegativitu. Třetí podmínka poskytuje poslední, N-tou, rovnici, kterou potřebujeme. \todo{jenomže rovnic je n+1}.

Nejenom z hlediska úspornosti zápisu, ale také pro pozdější počítačové řešení je vhodné tuto soustavu rovnic zapsat maticovým schématem

\(Aq=b\), kde \(q\) je vektor neznámých.

\[
\begin{pmatrix}
B_1 & \frac{\kappa}{R{1,2}} & \dots & \frac{\kappa}{R{1,N}} & -1 \\
\frac{\kappa}{R{2,1}} & B_2 & \dots & \frac{\kappa}{R{2,N}} & -1 \\
\vdots & \vdots & \ddots & \vdots & \vdots \\
\frac{\kappa}{R{N,1}} & \frac{\kappa}{R{N,2}} & \dots & B_N & -1 \\
1 & 1 & \dots & 1 & 0
\end{pmatrix}
\begin{pmatrix}
q_1 \\
q_2 \\
\vdots \\
q_N \\
\bar{\chi}
\end{pmatrix}
=
\begin{pmatrix}
-A_1\\
-A_2\\
\vdots \\
-A_N\\
Q\\
\end{pmatrix}
\]


Pro použití EEM potřebujeme znát prostorové uspořádání atomů v molekule, parametry a pro každý atom a konstantu kapa.

Potencialní energie

Můžeme značit \(E\), neboť jiná než potencialní energie v tomto odvození nevystupuje nebo V (jak je obvyklé v kvantové mechanice)

\[E=\frac{q_1 q_2}{\epsilon r_{1,2}}\]

Parametrizace

Parametry mají chemický význam

\section{Charge Equilibration (QEq)}

\[
\begin{pmatrix}
Q_1 & Q_2 & \dots & Q_i & -1 \\
J_{2,1} - J_{1,1} & J_{2,2} - J_{1,2} & \dots & J_{2,j} - J_{1,j} & -1 \\
\vdots & \vdots & \ddots & \vdots & \vdots \\
J_{i,1} - J_{1,1} & J_{i,2} - J_{1,2} & \dots & J_{i,j} - J_{1,j} & -1 \\
\end{pmatrix}
\begin{pmatrix}
-Q_{tot} \\
\chi_2-\xi_1 \\
\vdots \\
\chi_i-\chi_1 \\
\end{pmatrix}
=
-\begin{pmatrix}
-Q_{tot} \\
\chi_2-\xi_1 \\
\vdots \\
\chi_i-\chi_1 \\
\end{pmatrix}
\]

Stejně jako u metody EEM vycházíme z výrazu pro potencialní energii. Autoří QEq ve svém odvození nezabývají tím, jak tento vztah vypadá, ale 

že dE/dq

pro molekulu potom


Použítím vztahu pro ekvalizaci elektronegatity, což je 

Coulumbův zákon neplatí pro velmi malé vzálenosti, kdy už nelze atomy realisticky aproximovat jako bodové náboje. Proto  $q_1 q_2 J_{1,2}$, J je coulombovský integrál. Pro velké hodnoty R J = 1/R což je Coulumbův zákon a pro malé hodnoty se používá překryvový integrál pro Slaterovy orbitaly Jij = .

Parametrizace

Stejně jako u metody EEM mají parametry chemický význam.

\section{Split Charge equilibration (SQE)}

Tato sekce začíná popisem split charges a následně popisuje odvození modelu SQE způsobem v originálním článku.

Na rozdíl od předchozích metod uvažuje SQE i topologii molekuly.

Definice split charge

\todo{Split charge byl zaveden v metodě AACT.}

Náboj atomu vyjádříme jako jeho vlastní náboj a přesunutý náboj z jeho sousedů po kovalentních vazbách (\todo{v učebnici chemie na to je možná nějaká terminologie}) součet "dělených" nábojů na v Split charge.

\[Q_i = \sum_j overbar qij\]

Odvození metody Split Charge equilibration

V původním článku vycházejí z ekvilibrace náboje, na čež zavedou split charges podle definice nahoře, napíší původní vstah z jejich pomocí, v dalším kroku zavedou nové proměnné a ukáží, že při zavedení dalších tří podmínek je jejich konstrukce ekvivalentní s ekvilibrací náboje. Při přidání jiných podmínek přechází v metodu AACT. Tak je ukázáno, že metoda SQE (kde se dodatečné podmínky nevyskytují) je zobecněním obou těchto metod.

Princip ekvilibrace náboje, ekvivalentně ekvilibrace elektronegativity

\todo{proč je tady náboj to samý co chemický potenciál?, molární g energie?}

\chapter{Numerická matematika}

Numerická matematika studuje konstruktivní metody matematické analýzy a algebry. Konstruktivní metody jsou postupy, jejichž aplikací lze získat řešení matematických úloh s libovolnou pevně zvolenou přesností pomocí provedení konečného počtu aritmetických operací. Některé úlohy jsou řešitelné přímo (soustava linearních rovnic), jiné pouze přibližně (diferencialní rovnice). Přímá řešení jsou zatížena zaokrouhlovacími chybami, ke kterým dochází při reprezentaci čísel v počítači na omezeném počtu bitů. Přibližná řešení jsou navíc zatížena chybami metody, které vznikají tím, že se nehledá řešení původního problému, ale aproximace tohoto řešení. \citep[s.~147]{černá1987základy}

Pokud hromadění zaokrouhlovacích chyb během výpočtu může dramaticky ovlivnit výsledek, který získáme použitím numerické metody, je taková metoda numbericky nestabilní. Numerická stabilita je velmi žádoucí vlastnost numerických metod. \citep[s.~151]{černá1987základy}

Řešení soustav linearních rovnic

Jak metoda EEM tak Qeq nevyžaduje z výpočetního hlediska nic víc než vyřešení soustavy n linearních rovnic s n neznámými. Takovou soustavu můžeme maticově zapsat takto

\[Ax = b\]

Pro řešení soustav linearních rovnic nám numerická matematika nabízí velké množsví různých metod, jak přímých, tak aproximačních.

Násobení inverzní maticí

Pokud soustavu zprava vynásobíme maticí A^-1, dostaneme

\[A^{-1}Ax = A^{-1}b\]
\[x = A^{-1}b\]

Inverziní matici vypočteme jako

\[A^{-1} = \frac{1}{det A} A s vlnovkou\]

kde A s vlnovkou je matice adjungovaná k matici A.

Nevýhody řešení násobením inverzní maticí jsou jednak velký druhak aritmetických operací, které musíme provést a z toho plynoucí akumulace zaorkouhlovacích chyb, navíc je metoda na tyto chyby velmi citlivá pro soustavy, jejichž determinant je blízky nule, a nakonec metoda nedokáže využít strukturu zadání k paměťovým úsporám, protože inverzní matice je vždy hustá, a to i když vstupní matice byla symetrická nebo řídká.

Gaussova eliminace

Algoritmus, který se běžně vyučuje na přednáškách z linearní algebry. Numerická stabilita, změna pořadí řádků tak, aby. Tento algoritmus používají programy .... Je možno zpětně dosadit a vypočítat hodnotu. Na příkladech se ukázalo, že.

Dekompozice

Metody dekompozice. Výhodou je, že dekompozice řídké matice je opět řídká (zatímco inverzní matice bude hustá). Dekompozice má složitost n3, řešení se znalostí dekompozice má složitost n2. je numericky přesnější než násobení inverzní maticí.

Je možné vytvořit algirmus, který využívá, že matice je symetrická. To znamená, že se ušetří polovina paměťi.

Iterativní řešení

Interpolace

Pokud máme $n$ bodů, můžeme je proložit polynomem stupně $n+1$. polynomy se konstruují jako. Zajístí se tím, že každý z činitelů bude v bodě $x_i$ nabývat hodnoty $y_i$ a v ostatních bodech bude nabývat hodnoty 1. Nevýhoda je citlivost na chyby měření. Metoda interpoluje polynomem velkého stupně. Metodu nelze použít pokud máme více rozdílnych hodnot y pro jedno x_i.

\chapter{Strojové učení}

Linearní regrese

Minimalizuje součty čtverců svislých vzdáleností mezi přímkou a neměřenými body.

Alternativně je možné minimalizovat vzdálenosti mezi přímkou a naměřenými body.

Predikční chyba,

Nevýhoda metody je citlivost na odlehlé body.

Mohlo by se zdát, že metoda umí aproximovat pouze linearní závislosti. Není tomu tak. Například pokud chceme proložit parabolu, použijeme hypotézu x1, x2, kde x2 je x^2.

Metoda nejmenších čtverců

Strojové učení

Metody používané ve strojovém učení byly původně vyvinuty v matematické statistice (regrese), umělé inteligenci (, 

Učící data, testovací data a ověřovací data.
Učící data jsou vstupem pro parametrizaci hypotézy. Na testovacích datech se ověří tato parametrizace. Na ověřovacích datech se ověří volba hypotézy. Tento postup předchází situaci, kdy vývojář vyladí metodu pro jeden soubor dat, na kterém pak funguje výborně, ale zcela selhává na jiných datech.

RANSAC

Random sample consensus . Předpokládáme, že vstupní data obsahují malé množství odlehlých bodů. body inliers, správné hodnoty a outliers, chyby. Pokud bychom aplikovali metodu nejmenších čtverců na všechna data, bude výsledek negativně ovlivněn přítomností outliers. Metoda RANSAC provádí náhodné podmnožiny vstupních dat, vypočítá metodou nejmenších čtverců, . Běh metody selže, pokud v náhodném výběru ležely outliers. Iterace se mnohokrát opakuje. Vybírá ze vstupní množiny, za předpokladu, že budou le

Algoritmus byl poprve představen v článku z roku kde jej autoři využili při řešení problému v oblasti počítačového vidění.

Určit $p$, pravděpodobnost že algoritmus uspěje.

$w$ podíl správných bodů
$n$ počet bodů, které musíme vybrat
$k$ je počet iterací
$w^n$ je pravděpodobnost, že nevybereme odlehlý bod a $1-w^n$ je pravděpodobnost, že alespoň jeden z bodů je odlehlý bod
$(1-w^n)^k$ je pravděpodobnost, že běh algoritmu v žádné iteraci nevybere množinu platných bodů
\[1-p=(1-w^n)^k\]

Podmínkou fungování algoritmu je, že $w$, podíl správných bodů, je dostatečně vysoké. Z grafu je vidět, že pravděpodobnost zdárného běhu algoritmu rychle klesá s tím, jak se zvyšuje podíl odlehlých bodů v datech.

Protože uvádí, že chybných je jen několik málo molekul, zdálo by se, že metoda RANSAC by mohla být užitečná v první fázi běhu programu.




Zamyšlení

V této práci implementované metody, EEM a Qeq, 

Vědecká metoda, jak ji definoval anglický filozof Bacon, se uznačuje jako redukcionistická. Je možno ji popsat tak, že se snaží komplexní systém rozdělit na vzájemně co nejméně interagující oddělené části, prostudovat každou část zvlášť a na základě takto získaných zkušeností pochopit celek. Opakem redukcionistického přístupu je přístup holistický, který se snaží zkoumaný objekt pojmout jako celek. “Celek je víc než jen součet jeho částí”. Argumentem pro postoj, že samotný redukcionistický přístup je nedostatečný, je několik. Emergentní vlastnosti. Znám Austrálii, znám králiky, ale nikdo nedokázal včas odhadnout, co udělají králící v austrálii.

V součastnosti existuje kniha New kind of science, ve které Wolfram, který je znám jako zakladatel společnosti vyvíjející matematický softwarový balik Mathematica, a mezi studenty technických a přírodovědných oborů také pro webový portál wolframalpha.com

Popisuje jednoduchý “buněčný automat”, systém sestávající z řady osmi políček, která mohou být buď černá nebo bílá, a tabulky pravidel, která zadává “přechodovou funkci” mezi stavy automatu. Wolframův argument je následujicí: takto zadaný buněčný automat je velmi jednoduchý systém, stejnětak funkce řídící přechody mezi stavy. Přesto může automat vykazovat složité vzory aktivity, které přímo nevyplývají z přechodových pravidel.

V knize Fragments of Reality si autoři všímají ještě dalšího jevu. Přestože Jako příklad uvádějí experiment, kdy autoři vytvořili komplikovanou simulaci ekosystému, zahrnující xx parametrů, některé z nich dokonce náhodné, přesto bylo možno vývoj simulace popsat pomocí čtyř parametrů agregovaných přes všechny čtverce (ZDROJ) 

Autoři předkládají svoji představu, že následující strukturu. Z jednoduchých základních zákonů se objevuje složitý (autoři jej nazývají „noční můra redukcionistů” (reductionist nightmare)), který se následně opět redukuje na relativně předvídatelnou strukturu. Následně se autoři zamýšlejí, nakolik je (ve skutečnosti je svět složítý, jen si v něm lidské vědomí umí najít to jednoduché)
Poznatky

přesnost

doba kompilace

Použití knihovny se ukázalo nepříznívý efekt na dobu kompilace, která na počítači dosahovala nepřijemných 15 vteřin. Pomocí 

S knihovnou souvisí i další problém. Knihovna je založena na šablonách (možnost používat různé typy jako prvky matic) a preprocesorových makrech (dovolují autorům rozdělit definici třídy mezi více souborů -- hlavní definici a pluginy)


Zdroje

Ian Steward Why Beauty Is Truth
Popularizační kniha, která v historickém kontextu představuje myšlenky, které nakonec (v retrospektivním pohledu) vyvrcholily formulací teorie grup. Výklad se zaměřuje na osobnosti zůčastněných matematiků. Závěr knihy se zabývá matematickou fyzikou.

Essentials of Computational Chemistry
Učebnice počítačové chemie, předpokládá se znalost základů kvantové fyziky, podrobně popisuje principy stojící za kvantově mechanickými výpočty (zejména aproximace při řešení Schrodingerovy rovnice a v teorii funkcionálu hustoty). V úvodu se shrnují důležité velké koncepty  kvantové mechaniky, ujasňuje se například pojem orbitalu.

Elements of Physical Chemistry
Zkrácená verze masívní (20xx stran) učebnice Physical Chemistry od stejného autora, určená pro studenty biologických oborů.

Brdečka Fysikální chemie
Učebnice fyzikální chemie. Kvantové mechanice je věnována jen malá část knihy, která představuje záření černého tělesa, Bohrův model atomu a Schrodingerovu rovnici. V celé knize se vůbec nevyskytuje hamiltonián (operátor celkové energie) a Schredingerova rovnice se uvažuje vždy jen pouze jednoelektronová.

Ian Stewart, Jack Cohen Fragments of reality
Popularizační kniha, ve které autoři předkládají svůj pohled na otázky spadající do filozofie vědy.

RALPH G PEARSON Chemical hardness and density functional theory J. Chem. Sci., Vol. 117, No. 5, September 2005, pp. 369–377.
Článek, ve kterém jeden z objevitelů pojmu chemická tvrdost vysvětluje proces, kterým byl tento pojem zaveden a posléze kvantitativné definován. V dobře srozumitelném výkladu je dána do souvislosti ionizační energie, tvrdost a elektronegativita, což se hodí pro článek (o qeq)

SOftware

Open Babel
Eigen
QT4
QtTest
gcc

%Výsledkem této práce je 
%Ačkoli jsou atomové parcialní náboje v principu neměřitelný teoretický konstrukt, zůstanou i na dále důležitým prvkem v chemii. Do způsobů, jak je určovat bylo investováno mnoho výzkumného úsili.

%Polární molekuly

%Nepolární molekuly

%Molekuly léčiv




\chapter{Předchozí implementace}
EEM

Programy implementující parametrizaci musejí implementovat i řešení EEM systému.

práce na MU

včetně parametrizace

Tomáš Raček ve své diplomové práci implementoval program NEEMP ()

Bakalářská práce Jakuba Vaňka prezentuje nástroj EMP (EEM Method Parametrization). EMP 

Poslední nástroj nazvaný TRON vytvořil Tomáš Raděj.

Řešení EEM systému

TRON a EMP používají vlastní implementaci Gaussovy eliminace s pivotováním, NEEMP používá \todo{ale co je to za konkrétní operace} procedury ssysv, sgesv a sspsv  z knihovny Intel Math Kernel Library.

Parametrizace

Tron používá exhaustivní prohledávání intervalu kappa, NEEMP vychází se stejného přístupu, s tím, že v první fázi prostupně projde interval kappa řídčeji s experimentálně určenou konstatnou a následně takto experimentálně zjištěnou závislost lokálně aproximuje parabolami a snaží se cíleně prozkoumávat lokální maxima je postavený na Brentově metodě využívá GNU Scientific Library. \todo{ten NEEMP je asi/možná spatně popsané}

EMP a NEEMP vyřazují některé molekuly z učící sady, což vede k vylepšení výsledné korelace na celé sadě.

Programovací jazyk. TRON je napsán v jazyce Java, NEEMP v jazyce C.

Licence. K programu TRON jsou dostupné zdrojové kódy.

Paralelní provádění

Tron a \todo{EMP} pracují seriově. NEEMP používá OpenMP a matematická knihovna interně používá OpenMP a \todo{SIMD} Míru paralelizmu je možno volit.

NEEMP dospěl k závěru, že paralelní výpočet matice není vhodný a paralelizmus co se týče výpočtů pro ruzné hodnoty lambda

bez parametrizace

Radka Svobodová Vařeková 

další sw

openbabel

QEQ

openbabel
qtpie
Materials Studio
GULP



SQE

vše možné
qfit, cit verstraelen2011significance , není dostupný

Hypotéza

Metoda nejmenších čtverců je známá tím, že malé množství odlehlých bodů může značně ovlivnit výsledek. Je tedy možné, že v učící sadě se nacházejí molekuly, které dají vzniknout odlehlým bodům v příslušném kroku parametrizace. Tyto molekuly mohou být buď neobvyklé z hlediska distribuce náboje a námi používaný model není schopen správně postihnout jak tyto body, tak většinu molekul. Vynecháním ze sady dovolíme modelu správně se naparametrizovat na většinu, což vede k lepší korelaci, než pokus parametrizovat vše, ale nic správně. Další možnost je, že topologie těchto molekul je v databázi zadána chybně. Dále je možné, že data jsou jinak poškozená. Chybějící atomy a tak podobně. Metoda EEM je odvozena pro molekuly v globálním minimu nebo blízko něj \todo{je to pravda?}. Není možné vyloučit ani kombinaci všech zmíněných možností.

Metoda

Hypotézu o chybných vstupních datech je možno ověřit pomocí optimalizace vstupních molekul. Pokud se optimalizovaná konfigurace liší od vstupní, znamená to, že molekula byla zadána chybně. Optimalizace je iterativní proces. Pro urychlení této operace stačí provést pouze první iteraci optimalizačního procesu, což odhalí, zda byla molekula v lokálním minumu nebo ne.

Všechny hypotézy je možno otestovat manualní inspekcí vyřazených molekul.

\section{Parametrizace}

Při parametrizaci metod EEM i Qeq je cílem nalézt takové parametry, které minimalizují predikční chybu na učící sadě molekul. V případě metody EEM to znamená pro pevně dané $\kappa$ nalézt zbývající parametry metodou nejmenších čtverců.

Metoda nejmenších čtverců

predikční chyba a ta druhá, čtverce nebo průměty

Matrix computations Least square methods

je možno velmi kompaktně vyjádřit maticovým zápisem jako 

C = A + * b

kde A+ značí Moore-penrosova pseudoinverze matice A.

V případě, že na rychlosti nezáleží (při interaktivním počítání v Matlabu) je vhodné použít SVD dekompozici. Neklade žádné specialní požadavky na vstupní matici a je velmi přesná.

@misc{Sestrienková2013thesis,
AUTHOR = "SESTRIENKOVÁ, Simona",
TITLE = "Mooreova-Penrosova pseudoinverze, iterační algoritmy pro výpočet [online]",
YEAR = "2013 [cit. 2014-04-28]",
TYPE = "Bakalářská práce", 
SCHOOL = "Masarykova univerzita, Přírodovědecká fakulta",
SUPERVISOR = "Jiří Zelinka", 
%URL = "Dostupné z WWW <http://is.muni.cz/th/379436/prif_b/>",
}

http://vene.ro/blog/inverses-pseudoinverses-numerical-issues-speed-symmetry.html
\(\operatorname{arg\,min}_x ||b - Ax||\)

Je možno jednoduše odvodit, pokud vyjdeme z rovnice přímky y = ax+b, vyjádříme hodnotu b-Ax jako funkci v proměnných a,b a tuto kvadratickou funkci minimalizujeme nalezením derivace.

dovoluje přesunout problém výpočtu na zvolenou matematickou knihovnu. Tomáš Ráček ve své diplomové práci setkal s problémem nedostačujícího řádového rozsahu typu float32 při přímé implementaci vzorce (). Sumace kdy dochází k příčítání relativně velkého součtu v akumulátoru k v porovnání s ní malé hodnotě jednotlivých prvků vektoru.

Pseudoinverze

SVD, QR

% Linear Least Squares Problem
% classes.soe.ucsc.edu/cmps290c/Spring04/paps/lls.pdf‎
% ... among the three. On the other hand SVD is the slowest and most accurate. ... QR factorization transform the linear least square problem into a triangular least ...
% 
% PDF]
% NUMERICALLY EFFICIENT METHODS FOR SOLVING ...
% math.uchicago.edu/~may/REU2012/REUPapers/Lee.pdf‎
% by LEE DO Q - ‎2012 - ‎Related articles
% Aug 24, 2012 - The QR Factorization in Least Squares Problems. 10. 5.4. ... Singular Value Decomposition (SVD) and its robustness in solving rank-deficient.
% 
% [PDF]
% Lecture 9 - Least Squares, QR and SVD - University of ...
% https://wiki.engr.illinois.edu/download/attachments/.../lecture09.pdf?...‎
% Mar 15, 2011 - Lecture 9. Least Squares, QR and SVD. T. Gambill. Department of Computer Science. University of Illinois at Urbana-Champaign. March 15 ...
% 
% Eigen: Solving linear least squares systems
% eigen.tuxfamily.org/dox-devel/group__LeastSquares.html‎
% This page describes how to solve linear least squares systems using Eigen. ... page are the SVD decomposition, the QR decomposition and normal equations.
% 
% [PDF]
% Linear Algebra, part 3 QR and SVD Going back to least ...
% www.csc.kth.se/utbildning/kth/kurser/DN2266/matmod12/LA3_2p.pdf‎
% Linear Algebra, part 3. QR and SVD. Anna-Karin Tornberg. Mathematical Models, Analysis and Simulation. Fall semester, 2012. Going back to least squares.

\chapter{Implementace}

\section{Možnosti pro paralelizaci výpočtů}

Možnosti paralelizace

Můžeme úlohu rozdělit na na sobě nezávislé úkoly. Nyní se naskýtají dvě možnosti. Buď implementovat paralelizmus na úrovni úkolů a provádět jich souběžně více, nebo paralelizovat provádění každého úkolu zvlášť. Pokud zvolíme druhou možnost, budeme na vyřešení prvního úkolu čekat kratší dobu, ovšem vyřešení všech úkolů bude trvat déle, což je důsledek Amdalova zákona. První možnost nabízí lepší efektivitu využívání zdrojů. První možnost nemůžeme zvolit, pokud se větší množství úloh nevejde souběžně do paměti.

Vlákna a procesy

Na úrovni operačního systému. Rozdíl mezi vláknem a procesem je, že vlákna sdílejí paměť, vytvoření vlákna je rychlejší.

False sharing.

SIMD (Single Instruction Multiple Data)

Moderní procesory mají ve své instrukční sadě instrukce, které jsou schopné provádět operace nad více než jednou hodnotou najednou. Nevýhodou SIMD je nutnost tyto vektorové instrukce v programu používat explicitně. Existují kompilátory, které dovedou vektorizovat kód automaticky. Často je jim potřeba pomoci, blocking. Data musí být v paměti zarovnána.

SIMT (Single Instruction Multiple Thread)

Model programování grafických karet. Warp sestává z 32 vláken provádějících tu stejnou instrukci nad různými daty. Je zapotřebí mezi vlákny vhodným způsobem přistupovat k paměti. V případě, že dojde k divergenci, umí procesor vlákna automaticky maskovat.

Partition campling.

Přepínání vláken je implementováno v HW, je tedy mnohem rychlejší než na CPU, kde tuto operaci řídí plánovač OS.

Konflikty paměťových bank.

Utilization.

Obecně je možno říct, že CPU je vhodné pro úlohy kde je hodně komunikace.

Posix threads Pthreads

OpenMP

SMP, podporuje i SIMD. V programu se používá pomocí specialních \#pragma direktiv pro kompilátor. Při překladu je nutno zapnout podporu.

MPI

MPI je API pro zasílání zpráv mezi jinak nezávisle na sobě běžícími procesy, které mohou, ale nemusejí sdílet paměťový prostor. Podle toho, v jaké konfiguraci tyto procesy běží, může zasílání zprávy obnášet buď kopírování paměti v rámci počítače, nebo i síťovou komunikaci mezi počítači, a to skrze rozličná sítová rozhraní (iso xxx Ethernet, Infiniband). OpenMPI je vhodné pro použítí v asymetickém multiprocesingu, kdy je smysluplné rozlišovat mezi lokální, rychlou pamětí, a vzdálenou, pomalou pamětí. Právě přístup ke vzdálené paměti je řešen mechanizmem zasílání zpráv. V případě běhu na jednom počítači má smysl rozlišovat mezi lokální a vzdálenou pamětí u architektur NUMA (Nonuniform Memory Access), která rozděluje paměťový prostor mezi jednotlivé procesory a přístup k paměti příslušející jinému procesoru je pomalejší. Důvodem pro zavedení architektury NUMA jsou výsoké nároky na architekturu cache, které klade dřívě používané SMP.

The Intel® Trace Analyzer and Collector has a long-standing reputation as a profiler that helps you understand MPI application behavior, and effectively visualize bottlenecks in your code. 

Jedná se o běžnou knihovnu v jazyce C. Při překladu stačí přidat parametr pro linker.

Ve verzi 2 není možné za běhu přidávat a odebírat počítače.

Intel Thread Building Blocks

Intel Cilk Plus

% Should I expect Intel Cilk Plus to outperform TBB, OpenMP and MPI?
% 
% Intel Cilk Plus offers a competitive alternative for parallel programming that is far easier to use than MPI. It is intended to complement TBB and OpenMP, enabling programmers to parallelize applications that may be too complex or cumbersome to fit into one of these frameworks. Cilk Plus is not intended to be a replacement for these platforms, however. In general, one should not expect to see performance improvements by convert an existing parallel code in TBB or OpenMP to Cilk Plus. TBB, OpenMP and MPI continue to be good choices for writing High Performance Computing applications.
% 
% Intel Cilk Plus is designed to be a unique, general-purpose solution which provides good scalability for multi-core programs across a variety of applications and machines, and which allows programmers to exploit both data and task parallelism in their program in a straightforward, maintainable manner. The tradeoff is that Cilk Plus may not be as finely tuned for specific programming patterns or environments as TBB or OpenMP.
% 
% In general, each of the parallel programming frameworks has their own strengths and weaknesses. Which framework offers the best performance depends on both on the kind of application and how much the code has been tuned for a particular platform. One should not expect naïve conversions of tuned code from one platform to another to necessarily be comparable in performance, since each platform has its own unique tuning methodology and preferred programming style.



GPGPU

Využití grafických (geometry processing unit) k provádění obecných výpočtů. Fixed function pipeline a s cílem umožnit grafickým programátorům flexibilnější a zprístupnit vnitřní fungování karty. Nejprve probíhalo vytvářením specialních vertex a fragment shaderů, které pro určitá vstupní data generovaly obrázek, který bylo možno opět relativně snadno transformovat ve výsledek výpočtu. V reakci na tyto akademické snahy začali výrobci grafických karet pracovat na aplikačních rozhraních zpecialně pro GPGPU.

pozn pod čarou houghova (čti háfova) transformace fit, nalezení instancí parametrizovaného modelu v datech, známá zejména pro použítí ve zpracování obrazu k detenci přímek nebo kružnic.

CUDA

Cuda je výpočetní API vytvořené firmou Nvidia

OpenCL

Apple.

Kernely se kompilují ze zdrojového kódu až při spuštění programu na cílovém hardware. Z tohoto důvodu je nutné s programem distribuovat i zdrojové kódy a může dojít k úniku duševního vlastnictví tvůrce programu. Podobným problémem trpí i grafická knihovna OpenGL. Cuda i DirectX distribuují kernely zkompilované do platformně nezávislého mezikódu.

Cude nabízí dvě různá API. První, určené začátečníkům a pro situace, kde není nutné . Kernely se píší do stejných souborů, jako kód aplikace a na první pohled vypadají jako běžné funkce v jazyce C++. Ať už programátor použije libovolné API, nemusí řešit kompilaci, neboť tu obstará k tomu určený frontend nvcc, který rozdělí kódy kernelů a kód aplikace, a na každé zavolá příslušný kompilátor. OpenCL API je více nízkoúrovňové a nutí programátora zabývat se větším množstvím detailů, i když tuto flexibilitu nepotřebuje. Kompilace kernelů je manualní proces, programátor musí 

Překážkou při využívání dedikovaného hardware často je problém relativné nízké rychlosti sběrnice mezi tímto zařízením a operační pamětí počítače. je nutné, aby se na dedikovaném hardware vykonalo alespoň takové množství práce, které by ospravedlňilo časové náklady nuté k překopírování dat na grafickou kartu a následně výsledku zpět do hlavní paměti.

Výhodou OpenCL je, že existují implementace pro CPU i další specifický hardware jako jako FPGI (Field Programable, programovatelná hradlová pole). Kompilátor umí namapovat OpenCL kernel na grafické karty, paralelní architektury v procesorech (vícero procesorů v počítači, vícero jader v procesoru, SIMD) i FPGA (softwarem definovaný logický obvod). Díky tomu je možné jeden program spouštět na celé řadě zařízení. Heterogeneous computing.

Funkce přibývají pomaleji, tuning je stejně potřeba provádět pro každý hw zvlášť. Extenze. [to je ze slajdů]

https://aur.archlinux.org/packages/beignet/ intelí runtime pro grafiky na linux

--%v linuxu je nějaké Clover a Gallium3D \url{http://www.phoronix.com/scan.php?page=news_item&px=MTM1MzM}

%http://portablecl.org/

%https://fedoraproject.org/wiki/Changes/OpenCL

clpeak, Find peak OpenCL capacities like bandwidth \& compute 

AMD změnila architekturu grafických karet, bylo wliw

Nvidia, AMD (GPU i CPU), Intel, (FPGA) a další. Existují experimentální implementace WebCL pro prohlížeče Chrome, Firefox a dedikované javascriptové běhové prostředí NodeJS.

Z pohledu aplikace prezentuje OpenCL hierarchii objektů. Na nejvyšší úrovni je platforma. Pod každou platformou se může nacházet jedno nebo více zařízení. Aplikace může na jednom nebo několika zařízeních patřících do stejné platformy vytvořit kontext. 

Zařízení
work-group, work item
global memory
local memory

CUDA
global memory | global memory
local memory | shared memory
|registers

work group
work item
preferred *** size | warp

Globální paměť se nachází na výpočetním zařízení a před započetím výpočtu je nutné do ní překopírovat vstup a po skončení výpočtu naopak zpět překopírovat výstup. Mezi operační pamětí počítače a grafickou kartou se nachází sběrnice PCI-E, která je schopna přenášet 4 GB/s, řádově 200 

posuzování výkonu: occupancy, poměr maximální výpočetní a přenosové kapacity vůči využívané výpočetní a přenosové kapacitě, srovnání s dříve dosaženými výsledky v literatuře, profilery umožňují změřit occupancy

Renderscript

Jedná se o API nad CPU a GPU v operačním systému Android

Precision Memory Leak Detection Using the New On-Demand Leak Detection in Intel® Inspector XE

WebCL

Implementace OpenCL API v javascriptu, Nokia (Firefox), , Motorola Mobility (NodeJS) experimentální implementace.

existují OpenCl knihovny pro Python, ...

C++ , DirectX11compute

Rozšíření C++ které umožňuje programovat pro DirectX11 Compute v jazyce C++.

Communicating sequential programs (CSP)

Autorem paradigmatu CSP je Hoare. Podstatou CSP je strukturovat program jako vícero nezávislých procesů, které spolu komunikují pomocí zasílání zpráv přes kanály. Na rozdíl od modelu aktorů nemají tyto procesy oddělený paměťový prostor. Programování v tomto modelu je založeno na dodržování konvencí. Pokud program odešle referenci na měnitelný objekt pomocí kanálu jinému proceu, nesmí dotyčný objekt už sám nikterak měnit nebo číst. V situacích, kdy mechanizmus zpráv a kanálů nestačí, je možné využít klasícké programovací přístupy pomocí zámků a dalších primitiv. "Share memory by communication. Don't communicate by sharing memory." V jazyce C++ je tento přístup podporován pomocí knihoven. Programovací jazyk Go obsahuje CSP konstrukce přímo jako klíčová slova. Mezi C++ a Go existují možnosti pro interoperabilitu a je tudíž myslitelné vytvořit jádro aplikace v C++ a následně je spouštět paralelně z programu v jazyce Go.

Zvolené řešení

OpenCL + MPI, přímo i prostřednictvím knihoven, OpenMP pro plnění matic a výpočet párových vzdáleností.

Hvězda, jeden počítač čte vstup, plní matice a posílá...

\fixme{Ehm, paralelizace je vhodnější pro případnou parametrizaci, takže spíš nachystat to do stavu, aby se k tomu dala doprogramovat paralelní parametrizace}

Nástroje pro OpenCL

AMD

http://developer.amd.com/tools-and-sdks/heterogeneous-computing/codexl/
debugger, profiler a statický analyzátor pro OpenCL kernely

knihovny clMath (FFT a Blas), clMAGMA (umí i least squares)
Bolt, abstrakce nad OpenCL a C++ AMP

knihovny

ViennaCL

Knihovna tvoří nádstavbu nad  a navíc obsahuje vlastní implementované jako OpenCL kernely. Mezi těmito třemi implementacemi je možno přepínat.

Řešení soustavy linearních rovnic se MPI nehodí, jelikož se jedná o příliš malý problém a zasílání zpráv má příliš velkou režii, mohl by být ale užitečný z hlediska rozděnení zátěže mezi více počítačů.

Relativné časté je kombinování výpočtů na grafické kartě s OpenMPI, což umožňuje využít současně grafické karty na vícero počítačích.

Knihovna ViennaCL nabízí backendy pro OpenMP, OpenCL i Cuda. Navíc umí interoperabilitu s knihovnou Eigen, kterou jsem si vybral hned na začátku.

OpenMP využíváno jen jako k využívání vícero grafických karet na jednom systému. https://www.wiki.ed.ac.uk/display/ecdfwiki/Use+multiple+GPU+devices+with+OpenMP+and+CUDA streamy, to advanced api.

\section{Electronegativity equalization method}

\section{Charge equilibration}

Metodu Charge Equilibration (QEq) v článku Charge Equilibration for Molecular Dynamics Simulations \fixme{citace}.

\chapter{Použité sady molekul}

DTP NCI Developmental Therapeutics Program

%\url{http://dtp.nci.nih.gov/docs/3d_database/dis3d.html}

Milne, G.W.A., Nicklaus, M.C., Driscoll, J.S., Wang, S. and Zaharevitz. D. The NCI Drug Information System 3D Database. J. Chem. Inf. Comput. Sci. 34:1219-1224 (1994).

\chapter{Výpočet parcialních nábojů QM}

\section{MetaCentrum}

Národní Gridová Infrastruktura MetaCentrum je aktivita sdružení CESNET, z.s.p.o, která sdružuje akademickou výpočetní infrastrukturu v České Republice. Zaměstnancům a studentům zapojených institucí je umožněno zdarma využívat dostupnou infrastrukturu pro výzkumné účely.

Pro obsluhu výpočetních zdrojů je využíván dávkový systém TORQUE (Terascale Open-source Resource and QUEue Manager), což je reimplementace staršího systému PBS (Portable Batch System). Uživatelé zasílají výpočetní úlohy do fronty a systém je postupně spouští. Priorita uživatelů je dána jednak podle jejich pracoviště (uživatelé příslušející k organizaci, která vlastní určitý hardware, mají prioritu využívat tento hardware) a podle fairshare, což je číslo vyjadřující, jak moc uživatel systém v nedávné době využíval. Čím více uživatel propočítal, tím je toto číslo nižší, a tím menší prioritu jeho úlohy dostanou. Svoji fairshare si může uživatel zvýšit, když bud publikovat články s poděkováním MetaCentru.

Úlohu do systému TORQUE zašlete příkazem qsub.

qsub -l walltime=3h -l nodes=1:ppn=2 -l mem=1g -l scratch=500mb mujskript.sh

Právě uvedený příkaz požaduje spustit skript.sh na jednom počítači se dvěma procesory a 1 GiB operační paměti. Přepínač -l udává očekávanou dobu běhu úlohy. Požadované zdroje je potřeba zadat co nejpřesněji, ale raději více než méně. Pokud úloha překročí kterýkoli ze zadaných limitů, bude zabita. Pokud úloha požaduje zbytečně mnoho zdrojů, bude zbytečně dlouho čekat ve frontě. Scratch je název pro lokální disk pro uložení dat v průběhu výpočtu.

Prováděný skript by měl 



Metacentrum je 

\section{Gaussian 09}

HE THAT RISETH LATE MUST TROT ALL DAY.

                  -- BENJAMIN FRANKLIN

Gaussian je počítačový program pro provádění kvantově chemických výpočtů, který Je významný zejména tím, že má implementovánu velmi širokou škálu různých výpočetních metod. Název programu je odvozen od Gaussových křivek, které se používají k aproximaci bázových funkcí, ve kterých je vyjádřena vlnové funkce. Program je dále zajímavý tím, že je drahý, a jeho licenční podmínky jsou značně restriktivní. Akademická licence programu Gaussian stojí přibližně 2000 USD na dva roky a součástí podmínek používání programu je, že k němu nebude umožněn přístup autorům konkurenčního software. Také je zakázáno zveřejnovat benchmarky programu. 

Gaussian pracuje v dávkovém režimu. V souboru s příponou .com uvedeme, jaký výpočet chceme provést, a následně vstupní molekuly. Program Gaussian vytvoří soubor .log, ve kterém jsou informace o průběhu výpočtu, výsledek výpočtu, náhodně vybraný citát a statistiky o době běhu programu. Soubor .log je dobré vždy celý prohlédnout a ujasnit si význam všech zpráv v něm, aby se ověřilo, že při výpočtu nedošlo k chybě.

# HF/6-31G

Populační analýzu je možno zvolit parametrem Pop. Pokud žádnou hodnotu neuvedeme, použije se Pop=Minimal, což způsobí výpsání Müllikenových nábojů.

%http://cactus.nci.nih.gov/download/nci/

BLAS (Basic Linear Algebra Subprograms) je API specifikující často používané operace linearní algebry nad vektory a maticemi. Pro přehlednost je knihovna rozdělena na tři části. BLAS Level 1 Blas Level 2 a Blas Level 3 obsahuje procedury pro operace mezi dvěma maticemi. Nad primitivy obsažené v BLAS je postaven LAPACK (Linear Algebra PACKage). První implementace v jazyce FORTRAN77 Pro obě rozhraní existuje celá řada implementací, výrobci počítačových procesorů často dodávají vlastní implementace, ATLAS obsahuje benchmark, který automaticky volí nejlepší implementace pro hardware. Obsahují významnou součást benchmarků pro HPC.

Jména procedur reflektují datový typ (s, single precision, float32, d, double precision, float64)




Pro řešený problém je význačný následujícími dvěma aspekty. Za prvé řešení soustav linearních rovnic, které je ale z hlediska řešeného problému implementační detail a pro jeho řešení je zapotřebí znalostí numerické matematiky, který je nejlépe nechat knihovnám.

Přepínače kompilátoru

kompilátory mají celou řadu přepínačů, které ovlivňují rychlost výsledného kódu. Všechny přepínače nejsou dostupné u všech kompilátorů.

ffast-math

vypíná kompatibilitu s IEEE755 či co. Kompilátor například (a * a * a * a) je vyhodnocováné jako (((a * a) * a) *a), tři násobení, s přepínačem je možno optimalizovat jako (a*a) * (a*a), dvě násobení. Citace?

-O1,O2, O3, Os

harden source

https://security.stackexchange.com/questions/24444/what-is-the-most-hardened-set-of-options-for-gcc-compiling-c-c
-fstack-protector-all -Wstack-protector --param ssp-buffer-size=4 
%-­D_FORTIFY_SOURCE=2 ­O2



Hardware

Lenovo Thinkpad E530 (X86\_64)
mobil! tablet! ARM
MetaCentrum
Aisa

Software

Kompatibilitu programu jsem ověříl překladem na třech různých na sobě nezávisle vyvíjených překladačích

g++ GCC Gnu Compiler Collection
Clang
icpc, kompilátor vyvíjený firmou Intel, verze pro Linux je k dispozici zdarma k nekomerčnímu využití.
android ndk
https://pm.bsc.es/projects/mcxx mercuium source to source

KDevelop

dovození typu (type inference), což umožňuje používat konstrukci auto. Další vývojová prostředí, která by splňovala moje požadavky QTCreator, Microsoft Visual Studio, Eclipse.

OpenBabel

Boost program options

Eigen3
\todo{kopírování, přímý BLAS dovoluje provádět více operací inplace?}

formát souboru s parametry

na základě formátů používaných ostatními programy

řádkový
oddělovač řádku může být cokoli běžně používané

\fixme{řádek #! jména sloupců, more trouble than it's worth}
řádky začínající # jsou komentáře \fixme{dovolit koment na konci řádku?}

není možné vynechat hodnotu ve sloupci
pořadí sloupců je dané výpočetní metodou
oddělovačem je mezera a tabulátor (nebo unicode whitespace)

je shopno postihnout i vstup pro SQF, kde je nutno zadávat i parametry pro vazby

klíč hodnota hodnota hodnota ...

stejný klíč není dovoleno uvést vicekrát

parametr pro zvalidování vstupu


\chapter{Parametrizace}

EEM: pro fixní kappa najdeme další parametry metodou nejmenších čtverců. Programy TRON, EMP a * používají tuto metodu.

QEq: Program v článku kadantsev2013fast používá kombinaci globální a lokální optimalizace, konkrétně vlastní genetický algoritmus následovaný gradientní metodou (podle informací v dodatku se jednalo o steepest descent) pro doladění parametrů. Optimalizují se pouze parametry elektronegativita a tvrdost, ostatní zůstávají konstantní.

\chapter{Vyhodnocení}

Existuje rozpor ohledně toho, co se od metod jako EEM a QEq učekává. Jako chemická teorie, parametry vypovídají o chemických vlastnostech zkoumaných systémů, nebo regresní/metoda strojového učení, která vytváří a výstup slouží pro další regresní modely předpovídající chemické vlastnosti. Ve druhém případě se nezajímáme o chemický význam parametrů, ale o kvalitu predikcí.

V této práci hodnotím metody z hlediska jejich schopnosti reprodukovat QM výpočet. Výsledné náboje pro použití do regresního modelu, tedy nezáleží ani tak na absolutní shodě, jako spíš na korelaci. (Relativní posun se regresní model může naučit kompenzovat).

Databáze

Mnohé informatické vědní obory mají tradici zkoušení metod na de facto standardních sadách testovacích dat.

\todo{SQE článek má sadu molekul.}

Je výhodné testovat různé pokusy o řešení problému na těch stejných datech.

Referencni databaze jsou dulezite z nasledujicich duvodu, poskytuji moznost srovnat tuzne implementace otestovane ruznymi lidmi, je na nich mozne sledovat vyvoj a zlepsovani metod v prubehu casu. Dobra referencni databaze by mela byt snadno dostupna, dostatecne variablilni, pripadne se zamerovat na zvlaste slozite instance, ale nesmi opomenout ani ty bezne

Databaze se muze delit na ucici a testovaci data, pripadne si toto rozdeleni provede az pri testovani algortmu jako proccentualni cast, nebo pomoci foldu a podobne.

Physionet, street nnumbers, computer vision competitions, machine learning ma nektere notoricky zname, east west train,

Nevýhodou je že postupem času autoři se mohou začít soustředit na vylepšení výkonu na testovacích sadách, namísto reality. Znají svoji sadu velmi dobře po letech práce a dělají věci, které přinesou pokrok jen pro řešení sady, ne pro praktickou aplikaci. Problém přeučení.

U modelu muzeme preferovat vypocetni jednoduchost, coz je pripad chemoinformmatiky, kde boovykle ypracovavame velke mnozstvi molekul a je vyhodne co nejvice kandidatu co nejrzchleji zamitnout a soustredit se jen na tz perspektivni. V pocitacovem videni je zadouci bzt schopen zpracovavat data v realnem case, tj 30 az v nekterzch aplikacich 120 i vice snimku za sekundu. Klasifikator vztvorenz pomoci boosting jednoduchzch klasifikatoru a kaskady s fast reject

\chapter{Závěr}

Ukazuje se, že obě metody jsou v dobré shodě s ab-initio výpočtem. Metoda QEq dosahuje výrazně lepších výsledků než EEM za cenu jen mírného zvýšení časové náročnosti výpočtu.



%% Lists of tables and figures, glossary, etc.
%\printindex
%\printglossary
%\listoffigures
%\listoftables

%% Bibliography from citace.bib
%\bibliographystyle{plain}
\bibliographystyle{csplainnat}
\bibliography{charges}

%% Additional materials
\appendix

Návod k programu Quick

%% End of the whole document
\end{document}
